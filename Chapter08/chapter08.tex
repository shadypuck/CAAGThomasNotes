\documentclass[../main.tex]{subfiles}

\pagestyle{main}
\renewcommand{\chaptermark}[1]{\markboth{\chaptername\ \thechapter:\ #1}{}}
\setcounter{chapter}{7}

\begin{document}




\chapter{Hyperbolic Functions}
\section{Introduction}
\begin{itemize}
    \item \marginnote{6/24:}\textbf{Hyperbolic functions}: Certain combinations of $\e[x]$ and $\e[-x]$ that are used to solve certain engineering problems (the hanging cable) and are useful in connection with differential equations.
\end{itemize}



\section{Definitions and Identities}
\begin{itemize}
    \item Let
    \begin{align*}
        \cosh u &= \frac{1}{2}(\e[u]+\e[-u])&
        \sinh u &= \frac{1}{2}(\e[u]-\e[-u])
    \end{align*}
    \begin{itemize}
        \item These combinations of exponentials occur sufficiently frequently that we give a special name to them.
        \item Although the names may seem random, $\sinh u$ and $\cosh u$ do share many analogous properties with $\sin u$ and $\cos u$.
        \item Pronounced to rhyme with "gosh you" and as "cinch you," respectively.
    \end{itemize}
    \item Like $x=\cos u$ and $y=\sin u$ are associated with the point $(x,y)$ on the unit circle $x^2+y^2=1$, $x=\cosh u$ and $y=\sinh u$ are associated with the point $(x,y)$ on the unit hyperbola $x^2-y^2=1$.
    \begin{itemize}
        \item Note that $x=\cosh u$ and $y=\sinh u$ are associated with the \emph{right-hand} branch of the unit hyperbola.
        \item Also note that sine and cosine are sometimes referred to as the \textbf{circular functions}.
    \end{itemize}
    \item Analogous to sine and cosine, we have the identity
    \begin{equation*}
        \cosh^2u-\sinh^2u=1
    \end{equation*}
    \item We define the remaining hyperbolic trig functions as would be expected.
    \begin{align*}
        \tanh u &= \frac{\sinh u}{\cosh u} = \frac{\e[u]-\e[-u]}{\e[u]+\e[-u]}&
        \sech u &= \frac{1}{\cosh u} = \frac{2}{\e[u]+\e[-u]}\\
        \coth u &= \frac{\cosh u}{\sinh u} = \frac{\e[u]+\e[-u]}{\e[u]-\e[-u]}&
        \csch u &= \frac{1}{\sinh u} = \frac{2}{\e[u]-\e[-u]}
    \end{align*}
    \item Since $\cosh u+\sinh u=\e[u]$, we can replace any combination of exponentials with hyperbolic sines and cosines and vice versa.
    \item Note that the hyperbolic functions are \emph{not} periodic.
    \begin{itemize}
        \item This does mean, though, that they have more easily defined properties at infinity.
    \end{itemize}
    \item \dq{Practically all the circular trigonometric identities have hyperbolic analogies}{267}
\end{itemize}



\section{Derivatives and Integrals}\label{sss:8.3}
\begin{itemize}
    \item \marginnote{6/25:}Derivatives of the hyperbolic functions:
    \begin{align*}
        \dv{x}(\sinh u) &= \cosh u\, \dv{u}{x}&
        \dv{x}(\cosh u) &= \sinh u\, \dv{u}{x}\\
        \dv{x}(\tanh u) &= \sech^2 u\, \dv{u}{x}&
        \dv{x}(\sech u) &= -\sech u\tanh u\, \dv{u}{x}\\
        \dv{x}(\coth u) &= -\csch^2 u\, \dv{u}{x}&
        \dv{x}(\csch u) &= -\csch u\coth u\, \dv{u}{x}
    \end{align*}
    \begin{itemize}
        \item Note that these are almost exact analogs of the formulas for the corresponding circular functions, the exception being that the negative signs are not associated with the cofunctions but with the latter three.
    \end{itemize}
    \item We now introduce the hanging cable problem, deriving the differential equation that represents the condition for equilibrium of forces acting on a section $AP$ of a hanging cable (Figure \ref{fig:hangingAP}).
    \begin{figure}[h!]
        \centering
        \begin{tikzpicture}[
            scale=2,
            every node/.style={black}
        ]
            \footnotesize
            \draw [-stealth] (-0.9,0) -- (2,0) node[right]{$x$};
            \draw [-stealth] (0,-0.3) -- (0,1.8) node[above]{$y$};

            \draw [gax,thick,yshift=-0.4cm,xscale=1.25] plot [domain=-0.7:1.4,smooth] (\x,{0.5*(e^\x+e^(-\x))});

            \draw [ylx,very thick,-latex] (1,0.94) coordinate (P) -- (1.7,1.44) coordinate (A) node[fill=white,inner sep=2pt,above left]{$T$};
            \draw [ylx,thick,-latex] (P) -- node[below]{$T\cos\phi$} ++(0.7,0) coordinate (B);
            \draw [ylx,thick,-latex] (B) -- (A) node[below right]{$T\sin\phi$};
            \draw [ylx,very thick,-latex] (0,0.6) -- (-0.8,0.6) node[below right]{$H$};
            \foreach \x/\y in {0.1/0.6,0.37/0.64,0.63/0.73,0.9/0.87} {
                \draw [ylx,thick,-latex] (\x,\y) -- ++(0,-0.3);
            }

            \begin{scope}[on background layer]
                \pic[draw,pic text={$\phi$},angle radius=6mm,angle eccentricity=1.3]{angle=B--P--A};
            \end{scope}

            \node [circle,fill,inner sep=1.5pt,label={above left:$P(x,y)$}] at (P) {};
            \node [circle,fill,inner sep=1.5pt,label={above left:$A$}] at (0,0.6) {};
            \node [below left] {$O$};
            \node at (0.5,0.3) {$W$};

            \draw [white,double=black,double distance=0.4pt] (0.9,0.88) rectangle (1.1,1);
            \draw [thin,->] (1,0.88) -- (1,0.5) -- (3,0.5);
            \begin{scope}[xshift=3cm]
                \normalsize
                \draw (0,0) rectangle (2.7,1.5);
                \clip (0,0) rectangle (2.7,1.5);
                \draw [gax,ultra thick] (0,0) -- (2,1.5);

                \draw [ylx,line width=2pt] (1,0.75) coordinate (a) -- node[below]{$\dd x$} (1.6,0.75) coordinate (b) -- node[right]{$\dd y$} (1.6,1.2) coordinate (c) -- node[above,xshift=-1mm]{$\dd s$} (1,0.75) pic[pic text={$\phi$},angle eccentricity=1.3]{angle=b--a--c};

                \node [circle,fill,inner sep=2.2pt,label={above left:$P(x,y)$}] at (1,0.75) {};

                \draw (0,0) rectangle (2.7,1.5);
            \end{scope}
        \end{tikzpicture}
        \caption{A section $AP$ of a hanging cable.}
        \label{fig:hangingAP}
    \end{figure}
    \item Let point $A$ to be the lowest point in the arc of the hanging cable, and let it be at $(0,y_0)$ in the Cartesian plane.
    \item Continue along the right arc of the cable until arriving at some point $P(x,y)$.
    \item We wish to consider only segment $AP$, so we need to anchor points $A$ and $P$ as if the rest of the cable were still there. Now every infinitesimal sliver of the cable is being pulled (downward) slightly by gravity, but significantly (tangentially) by the rest of the cable. Thus, we can compensate at point $A$ by pulling it tangentially left with some force $H$, and at point $P$ by pulling it tangentially up and to the right with some force $T$.
    \item Since the cable is at equilibrium, the three forces acting on the cable as a whole ($T$, $H$, and $W$) are balanced. Thus,
    \begin{align*}
        T\sin\phi &= W\\
        T\cos\phi &= H
    \end{align*}
    \item Combining these two equations gives an important result:
    \begin{align*}
        \frac{T\sin\phi}{T\cos\phi} &= \frac{W}{H}\\
        \tan\phi &= \frac{W}{H}
    \end{align*}
    \item Now $\tan\phi$ is a particularly important piece of the puzzle, because it actually equals $\dv{y}{x}$ (see the zoomed-in section of Figure \ref{fig:hangingAP}). Thus,
    \begin{equation*}
        \dv{y}{x} = \frac{W}{H}
    \end{equation*}
    \item Contrary to how it may look, $W$ is actually not a constant --- the weight of section $AP$ is dependent on $P$ (i.e., is dependent on how long the section is). If we assume that the cable has a uniform weight per length ratio $w$ and that section $AP$ is $s$ units long, then we have $W=ws$. Thus,
    \begin{equation*}
        \dv{y}{x} = \frac{ws}{H}
    \end{equation*}
    \item $s$ is just arc length. Thus, $s=\int_A^P\sqrt{1+(\dv*{y}{x})^2}\dd{x}$. However, because we cannot have an integral in a differential equation, we differentiate to find $\dd s = \sqrt{1+(\dv*{y}{x})^2}\dd{x}$.
    \begin{itemize}
        \item Note that this expression for $\dd s$ makes sense because, by the zoomed-in section of Figure \ref{fig:hangingAP}, $\dd s = \sqrt{\dd x^2+\dd y^2} = \sqrt{(\dd x^2/\dd x^2+\dd y^2/\dd x^2)\dd x^2} = \sqrt{1+(\dv*{y}{x})^2}\dd{x}$.
    \end{itemize}
    \item If we differentiate $\dv{y}{x} = \frac{ws}{H}$, we can get $\dd s$ into the equation and substitute, as follows, to yield the final differential equation.
    \begin{align*}
        \dv[2]{y}{x} &= \frac{w}{H}\dv{s}{x}\\
        \Aboxed{\dv[2]{y}{x} &= \frac{w}{H}\sqrt{1+\left( \dv{y}{x} \right)^2}}
    \end{align*}
\end{itemize}



\section{Geometric Meaning of the Hyperbolic Radian}
\begin{itemize}
    \begin{figure}[h!]
        \centering
        \begin{subfigure}[b]{0.48\linewidth}
            \centering
            \begin{tikzpicture}[scale=2]
                \footnotesize
                \coordinate (O) at (0,0);
                \coordinate (A) at (1,0);
                \coordinate (P) at (60:1);

                \node [below left] at (O) {$O$};
                \node [above right=-1mm,yshift=2mm,black] at (A) {$A(1,0)$};
                \node [above right=-1mm,black] at (P) {$P(x,y)$};

                \pic[fill=yly,angle radius=1cm,scale=2,pic text={$\theta$}]{angle=A--O--P};
    
                \draw [-stealth] (-1.5,0) -- (1.5,0) node[right]{$x$};
                \draw [-stealth] (0,-1.5) -- (0,1.5) node[above]{$y$};
                \draw circle (1cm);
            \end{tikzpicture}
            \caption{Circular radians.}
            \label{fig:hyperbolicGeometrica}
        \end{subfigure}
        \begin{subfigure}[b]{0.48\linewidth}
            \centering
            \begin{tikzpicture}[scale=2]
                \footnotesize
                \fill [yly] (0,0) -- (1.8,0) -- (1.8,1.5) -- cycle;
                \fill [ylz] plot [domain=0:1.19] ({0.5*(e^\x+e^(-\x))},{0.5*(e^\x-e^(-\x))}) -- (0,0);
    
                \draw [very thin] (2.4,1.5) -- (1.8,1.5) -- (1.8,-0.8);
                \draw [thin,<->] (0,-0.6) -- node[near start,xshift=2mm,fill=white,inner sep=1.5pt]{$x=\cosh u$} (1.8,-0.6);
                \draw [thin,<->] (2.2,1.5) -- node[fill=white,inner sep=1.5pt]{$y=\sinh u$} (2.2,0);
    
                \draw [-stealth] (-0.4,0) -- (2.5,0) node[right]{$x$};
                \draw [-stealth] (0,-1.5) -- (0,1.5) node[above]{$y$};

                \draw [ylx,thick] plot [domain=-1:1.3,smooth] ({0.5*(e^\x+e^(-\x))},{0.5*(e^\x-e^(-\x))});
                \node [below left] {$O$};
                \node [below left] at (1,0) {$A(1,0)$};
                \node [above left] at (1.8,1.5) {$P(x,y)$};
                \node [below right] at (1.8,0) {$Q$};
                \node [above right] at (1.54,-1.18) {$x^2-y^2=1$};
            \end{tikzpicture}
            \caption{Hyperbolic radians.}
            \label{fig:hyperbolicGeometricb}
        \end{subfigure}
        \caption{Geometric meaning of radians.}
        \label{fig:hyperbolicGeometric}
    \end{figure}
    \item For circular sine and cosine, the \dq{meaning of the variable $\theta$ in the equations $x=\cos\theta$, $y=\sin\theta$ as they relatee to the point $P(x,y)$ on the unit circle $x^2+y^2=1$ [is] the radian measure of the angle $AOP$ in [Figure \ref{fig:hyperbolicGeometrica}], that is $\theta = \frac{\text{arc }AP}{\text{radius }OA}$}{271}
    \begin{itemize}
        \item However, since $A=\frac{1}{2}r^2\theta=\frac{\theta}{2}$ for $r=1$, $\theta$ also equals twice the area of the sector $AOP$.
    \end{itemize}
    \item To understand the meaning of the variable $u$, calculate the area of the sector $AOP$ in Figure \ref{fig:hyperbolicGeometricb} as an analog to circular area.
    \begin{align*}
        A_{AOP} &= A_{OQP}-A_{AQP}\\
        &= \frac{1}{2}bh - \int_A^P y\dd{x}\\
        \intertext{\centering$
            \begin{aligned}
                y &= \sinh u,&
                x = \cosh u \Rightarrow \dv{x}{u} = \sinh u \Rightarrow \dd x &= \sinh u\dd{u}
            \end{aligned}
        $}
        &= \frac{1}{2}xy - \int_A^P \sinh^2u\dd{u}\\
        &= \frac{1}{2}\cosh u\sinh u - \frac{1}{2}\int_A^P (\cosh 2u-1)\dd{u}\\
        &= \frac{1}{2}\sinh u\cosh u - \frac{1}{2}\left[ \frac{1}{2}\sinh 2u-u \right]_{A(u=0)}^{P(u=u)}\\
        &= \frac{1}{2}\sinh u\cosh u - \left( \frac{1}{4}\sinh 2u-\frac{1}{2}u \right)\\
        &= \frac{1}{2}\sinh u\cosh u - \left( \frac{1}{2}\sinh u\cosh u-\frac{1}{2}u \right)\\
        &= \frac{1}{2}u
    \end{align*}
    \begin{itemize}
        \item This implies that $u$ also equals twice the area of the sector $AOP$ (the hyperbolic sector, that is).
        \item This means, for example, that \dq{$\cosh 2$ and $\sinh 2$ may be interpreted as the coordinates of $P$ when the area of the sector $AOP$ is just equal to the area of a square having $OA$ as side}{272}
    \end{itemize}
\end{itemize}



\section{The Inverse Hyperbolic Functions}
\begin{itemize}
    \item \marginnote{6/26:}The inverse of $x=\sinh y$ is $y=\sinh^{-1} x$.
    \begin{itemize}
        \item Since there is a one-to-one correspondence between $x$ and $y$ values for the inverse hyperbolic sine function, there is no need to define a principal branch (as there was with circular sine).
    \end{itemize}
    \item The inverse of $x=\cosh y$ is $y=\cosh^{-1} x$, where $y\geq 0$ and $x\geq 1$.
    \begin{itemize}
        \item Since there is a two-to-one correspondence between $x$ and $y$ values this time, we choose the positive values to be the principal branch and let the negative values be defined by the function $y=-\cosh^{-1} x$.
    \end{itemize}
    \item Note that the only other inverse hyperbolic trig function that needs a principal branch is (rather appropriately) $\sech x$. Likewise, the positive values make up the principal branch.
    \item Like the hyperbolic trig functions have exponential forms, the inverse hyperbolic trig functions have logarithmic forms.
    \begin{itemize}
        \item For example,
        \begin{align*}
            y &= \sinh^{-1} x\\
            \sinh y &= x\\
            \frac{1}{2}(\e[y]-\e[-y]) &= x\\
            \e[y]-\frac{1}{\e[y]} &= 2x\\
            \e[2y]-1 &= 2x\e[y]\\
            0 &= \e[2y]-2x\e[y]-1\\
            \e[y] &= x\pm\sqrt{x^2+1}\\
            y &= \ln\left( x+\sqrt{x^2+1} \right)
        \end{align*}
    \end{itemize}
    \item Like the inverse circular trig functions, the inverse hyperbolic functions are quite useful as the end results of integration of radicals. First, however, we must derive their derivatives.
    \begin{align*}
        \dv{x}\left( \sinh^{-1}u \right) &= \frac{1}{\sqrt{1+u^2}}\, \dv{u}{x}&
        \dv{x}\left( \cosh^{-1}u \right) &= \frac{1}{\sqrt{u^2-1}}\, \dv{u}{x}\\
        \dv{x}\left( \tanh^{-1}u \right) &= \frac{1}{1-u^2}\, \dv{u}{x},\quad|u|<1&
        \dv{x}\left( \sech^{-1}u \right) &= \frac{-1}{u\sqrt{1-u^2}}\, \dv{u}{x}\\
        \dv{x}\left( \coth^{-1}u \right) &= \frac{1}{1-u^2}\, \dv{u}{x},\quad|u|>1&
        \dv{x}\left( \csch^{-1}u \right) &= \frac{-1}{|u|\sqrt{1+u^2}}\, \dv{u}{x}
    \end{align*}
\end{itemize}



\section{The Hanging Cable}
\begin{itemize}
    \item We seek to derive the solution to the following differential equation associated with the hanging cable problem, as described in Section \ref{sss:8.3}.
    \begin{equation*}
        \dv[2]{y}{x} = \frac{w}{H}\sqrt{1+\left( \dv{y}{x} \right)^2}
    \end{equation*}
    \item Since the above equation involves the second derivative, we will have to deal with two constants of integration. Since it doesn't matter where the hanging cable lies in the Cartesian plane, we can choose its location such that the final answer will be as simple as possible.
    \begin{itemize}
        \item \dq{By choosing the $y$-axis to be the vertical line through the lowest point of the cable, one condition becomes $\dv{y}{x}=0$ when $x=0$}{277}
        \item \dq{Then we may still move the $x$-axis up or down to suit our convenience. That is, we let $y=y_0$ when $x=0$, and we may choose $y_0$ so as to give us the simplest form in our final answer}{277}
    \end{itemize}
    \item Let's begin solving the original equation. Since it involves $y'$ and $y''$ but not $y$, let $y'=p$ and start by integrating with respect to $p$.
    \begin{align*}
        \dv{p}{x} &= \frac{w}{H}\sqrt{1+p^2}\\
        \frac{\dd p}{\sqrt{1+p^2}} &= \frac{w}{H}\dd{x}\\
        \int\frac{\dd p}{\sqrt{1+p^2}} &= \int\frac{w}{H}\dd{x}\\
        \sinh^{-1}p &= \frac{w}{H}x+C_1
    \end{align*}
    \item Since $p=\dv{y}{x}=0$ and $x=0$, $C_1=0$. Thus,
    \begin{align*}
        \dv{y}{x} &= \sinh\left( \frac{w}{H}x \right)\\
        \dd y &= \sinh\left( \frac{w}{H}x \right)\dd{x}\\
        \int\dd y &= \int\sinh\left( \frac{w}{H}x \right)\dd{x}\\
        y &= \frac{H}{w}\cosh\left( \frac{w}{H}x \right)+C_2
    \end{align*}
    \item Since $y=y_0$ when $x=0$, $C_2=y_0-\frac{H}{w}$. Thus, let $y_0=\frac{H}{w}$. Therefore, we are finished:
    \begin{equation*}
        y = \frac{H}{w}\cosh\left( \frac{w}{H}x \right)
    \end{equation*}
\end{itemize}




\end{document}