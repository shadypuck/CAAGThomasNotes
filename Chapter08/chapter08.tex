\documentclass[../main.tex]{subfiles}

\pagestyle{main}
\renewcommand{\chaptermark}[1]{\markboth{\chaptername\ \thechapter:\ #1}{}}
\setcounter{chapter}{7}

\begin{document}




\chapter{Hyperbolic Functions}
\section{Introduction}
\begin{itemize}
    \item \marginnote{6/24:}\textbf{Hyperbolic functions}: Certain combinations of $\e[x]$ and $\e[-x]$ that are used to solve certain engineering problems (the hanging cable) and are useful in connection with differential equations.
\end{itemize}



\section{Definitions and Identities}
\begin{itemize}
    \item Let
    \begin{align*}
        \cosh u &= \frac{1}{2}(\e[u]+\e[-u])&
        \sinh u &= \frac{1}{2}(\e[u]-\e[-u])
    \end{align*}
    \begin{itemize}
        \item These combinations of exponentials occur sufficiently frequently that we give a special name to them.
        \item Although the names may seem random, $\sinh u$ and $\cosh u$ do share many analogous properties with $\sin u$ and $\cos u$.
        \item Pronounced to rhyme with "gosh you" and as "cinch you," respectively.
    \end{itemize}
    \item Like $x=\cos u$ and $y=\sin u$ are associated with the point $(x,y)$ on the unit circle $x^2+y^2=1$, $x=\cosh u$ and $y=\sinh u$ are associated with the point $(x,y)$ on the unit hyperbola $x^2-y^2=1$.
    \begin{itemize}
        \item Note that $x=\cosh u$ and $y=\sinh u$ are associated with the \emph{right-hand} branch of the unit hyperbola.
        \item Also note that sine and cosine are sometimes referred to as the \textbf{circular functions}.
    \end{itemize}
    \item Analogous to sine and cosine, we have the identity
    \begin{equation*}
        \cosh^2u-\sinh^2u=1
    \end{equation*}
    \item We define the remaining hyperbolic trig functions as would be expected.
    \begin{align*}
        \tanh u &= \frac{\sinh u}{\cosh u} = \frac{\e[u]-\e[-u]}{\e[u]+\e[-u]}&
        \sech u &= \frac{1}{\cosh u} = \frac{2}{\e[u]+\e[-u]}\\
        \coth u &= \frac{\cosh u}{\sinh u} = \frac{\e[u]+\e[-u]}{\e[u]-\e[-u]}&
        \csch u &= \frac{1}{\sinh u} = \frac{2}{\e[u]-\e[-u]}
    \end{align*}
    \item Since $\cosh u+\sinh u=\e[u]$, we can replace any combination of exponentials with hyperbolic sines and cosines and vice versa.
    \item Note that the hyperbolic functions are \emph{not} periodic.
    \begin{itemize}
        \item This does mean, though, that they have more easily defined properties at infinity.
    \end{itemize}
    \item \dq{Practically all the circular trigonometric identities have hyperbolic analogies}{267}
\end{itemize}




\end{document}