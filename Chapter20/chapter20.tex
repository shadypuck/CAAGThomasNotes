\documentclass[../main.tex]{subfiles}

\pagestyle{main}
\renewcommand{\chaptermark}[1]{\markboth{\chaptername\ \thechapter:\ #1}{}}
\setcounter{chapter}{19}

\begin{document}




\chapter{Differential Equations}\label{cht:20}
\section{Introduction}
\begin{itemize}
    \item \marginnote{9/9:}\textbf{Differential equation}: An equation that involves one or more derivatives, or differentials.
    \item \textbf{Type} (of a differential equation): A differential equation is either an \textbf{ordinary differential equation} or a \textbf{partial differential equation}.
    \item \textbf{Order} (of a differential equation): The order of the highest-order derivative that occurs in the equation.
    \item \textbf{Degree} (of a differential equation): The exponent of the highest power of the highest-order derivative, after the equation has been cleared of fractions and radicals in the dependent variable and its derivatives.
    \item \textbf{Ordinary differential equation}: A differential equation where the only derivatives that appear are those of a dependent variable $y$ varying as a function of a single independent variable $x$.
    \item \textbf{Partial differential equation}: A differential equation where partial derivatives appear.
    \item \dq{For example,
    \begin{equation*}
        \left( \dv[3]{y}{x} \right)^2+\left( \dv[2]{y}{x} \right)^5+\frac{y}{x^2+1} = \e[x]
    \end{equation*}
    is an ordinary differential equation, of order three and degree two}{691}
    \item \textcite{bib:Thomas} does not include a systematic treatment of partial differential equations, so he recommends Chapter 10 of \textcite{bib:Kaplan}.
    \item If $A,B,C$ are radioactive substances such that $A$ decomposes into $B$ at a rate proportional to the amount of $A$ present (proportionality constant $k_1$), $B$ decomposes into $C$ at a rate proportional to the amount of $A$ present (proportionality constant $k_2$), and $C$ decomposes into $A$ at a rate proportional to the amount of $C$ present (proportionality constant $k_3$), we can write the following system of differential equations.
    \begin{align*}
        \dv{x}{t} &= -k_1x+k_3z&
        \dv{y}{t} &= k_1x-k_2y&
        \dv{z}{t} &= k_2y-k_3z
    \end{align*}
    \begin{itemize}
        \item Since $\dv*{x}{t}+\dv*{y}{t}+\dv*{z}{t}=0$ in this case, our solution is that $x+y+z=C$, i.e., the statement that if the amounts of substances change in this manner, the total amount of substance present remains constant.
    \end{itemize}
\end{itemize}



\section{Solutions}
\begin{itemize}
    \item \textbf{Solution} (of a differential equation): A function $y=f(x)$ such that the differential equation is identically satisfied when $y$ and its derivatives are replaced throughout by $f(x)$ and its corresponding derivatives.
    \item Differential equations often have solutions in which certain arbitrary constants occur.
    \begin{itemize}
        \item However, these constants can often be resolved into specific values with the help of initial conditions.
        \item In fact, \dq{a differential equation of order $n$ will in general possess a solution involving $n$ arbitrary constants}{693}
        \begin{itemize}
            \item There is a more precise theorem that implies this result, but \textcite{bib:Thomas} neither states nor proves it.
        \end{itemize}
    \end{itemize}
    \item \textbf{General solution}: A solution that still contains all arbitrary constants arising from the solving process.
    \item Since finding general solutions requires calculus and finding specific solutions from general solutions and initial conditions only requires algebra, we will focus on finding general solutions.
    \item Note that this is only an introduction; for a more exhaustive treatment of differential equations, refer to \textcite{bib:MartinReissner}.
\end{itemize}



\section{First-Order Equations with Variables Separable}
\begin{itemize}
    \item If it is possible to collect all $y$-terms with $\dd{y}$ and all $x$-terms with $\dd{x}$, i.e., if it is possible to write the equation in the form
    \begin{equation*}
        f(y)\dd{y}+g(x)\dd{x} = 0
    \end{equation*}
    then the general solution is
    \begin{equation*}
        \int f(y)\dd{y}+\int g(x)\dd{x} = C
    \end{equation*}
    where $C$ is an arbitrary constant.
\end{itemize}



\section{First-Order Homogeneous Equations}
\begin{itemize}
    \item \textbf{Homogeneous} (differential equation): A differential equation that can be put into the form
    \begin{equation*}
        \dv{y}{x} = F(y/x)
    \end{equation*}
    where $F(y/x)$ is some function of $y/x$.
    \item Solving a first-order homogeneous differential equation.
    \begin{itemize}
        \item We use $u$-substitution.
        \item If we let $u=y/x$, then $\dv*{y}{x}=u+x\dv*{u}{x}$ by the product rule.
        \item Thus, the differential equation can be rewritten in the form $u+x\dv*{u}{x}=F(u)$, which can be solved in terms of $u$ and $x$ via separation of variables as follows.
        \begin{equation*}
            \frac{\dd{x}}{x}+\frac{\dd{u}}{u-F(u)} = 0
        \end{equation*}
        \item Once the above is solved, we can return the substitution to obtain our final solution.
    \end{itemize}
    \item \dq{Show that the equation $(x^2+y^2)\dd{x}+2xy\dd{y}=0$ is homogeneous, and solve it}{694}
    \begin{itemize}
        \item Via basic algebra, we can rewrite the above in the form
        \begin{align*}
            \dv{y}{x} &= -\frac{x^2+y^2}{2xy}\\
            &= -\frac{1+(y/x)^2}{2(y/x)}
        \end{align*}
        implying that it is homogeneous with $F(u)=-(1+u^2)/(2u)$.
        \item Therefore, the only remaining task is to solve
        \begin{equation*}
            \frac{\dd{x}}{x}+\frac{\dd{u}}{u-[-(1+u^2)/(2u)]} = 0
        \end{equation*}
        via separation of variables integration.
        \item After doing so, we obtain
        \begin{align*}
            \ln|x|+\frac{1}{3}\ln(1+3u^2) &= \frac{1}{3}\ln C\\
            x^3(1+3u^2) &= \pm C
        \end{align*}
        \item Therefore, returning our substitution, we have that
        \begin{equation*}
            x(x^2+3y^2) = C
        \end{equation*}
        is our solution.
    \end{itemize}
\end{itemize}



\section{First-Order Linear Equations}
\begin{itemize}
    \item \textbf{Degree} (of a term in a differential equation): The sum of the exponents of the dependent variable and any of its derivatives in a given term.
    \begin{itemize}
        \item For example, $(\dv*[2]{y}{x})$ is of degree one but $y(\dv*{y}{x})$ is of degree two.
    \end{itemize}
    \item \textbf{Linear differential equation}: A differential equation such that every term is of degree zero or degree one.
    \begin{equation*}
        \dv{y}{x}+Py = Q
    \end{equation*}
    \begin{itemize}
        \item A linear differential equation of first order can always be put into the above form, where $P,Q$ are functions of $x$.
    \end{itemize}
    \item \textbf{Integrating factor}: A function $\rho$ of the independent variable $x$ such that if the differential equation at hand is multiplied by $\rho$, it will compress into a form that is easier to integrate.
    \begin{itemize}
        \item For first-order linear differential equations, the function $\rho$ that we seek makes the left-hand side becomes the derivative of the product $\rho y$.
    \end{itemize}
    \item Deriving $\rho$ in terms of the values given in a general first-order linear differential equation.
    \begin{itemize}
        \item Multiplying by $\rho$, we have
        \begin{equation*}
            \rho\dv{y}{x}+\rho Py = \rho Q
        \end{equation*}
        \item Thus, since we want
        \begin{equation*}
            \rho\dv{y}{x}+\rho Py = \dv{x}(\rho y) = \rho\dv{y}{x}+\dv{\rho}{x}y
        \end{equation*}
        we must have by comparing terms that
        \begin{equation*}
            \dv{\rho}{x} = \rho P
        \end{equation*}
        \item It follows by separation of variables integration that
        \begin{align*}
            \frac{\dd{\rho}}{\rho} &= P\dd{x}\\
            \ln|\rho| &= \int P\dd{x}+\ln C\\
            \rho &= \pm C\e[\int P\dd{x}]\\
            \rho &= \e[\int P\dd{x}]
        \end{align*}
        \begin{itemize}
            \item Note that we can choose $\pm C=1$ since in the equation $\rho\dv*{y}{x}+\rho Py=\rho Q$, any $C$ term can be divided out of both sides anyways.
        \end{itemize}
    \end{itemize}
    \item With the help of the integrating factor, we can now derive the general solution to a first-order linear differential equation as follows.
    \begin{align*}
        \dv{y}{x}+Py &= Q\\
        \rho\dv{y}{x}+\rho Py &= \rho Q\\
        \dv{x}(\rho y) &= \rho Q\\
        \rho y &= \int\rho Q\dd{x}+C\\
        y &= \frac{1}{\e[\int P\dd{x}]}\left( \int\e[\int P\dd{x}]Q\dd{x}+C \right)
    \end{align*}
    \item Note that a first-order linear differential equation may also be separable, or homogeneous. In such cases, we have a choice of solution methods.
\end{itemize}



\section{First-Order Equations With Exact Differentials}
\begin{itemize}
    \item Refer to Section 15.13 for the method of solving exact differentials.
    \item Every first-order differential equation $P(x,y)\dd{x}+Q(x,y)\dd{y}=0$ can be made exact by multiplication by an integrating factor $\rho(x,y)$ having the property that
    \begin{equation*}
        \pdv{y}[\rho(x,y)P(x,y)] = \pdv{x}[\rho(x,y)Q(x,y)]
    \end{equation*}
    \begin{itemize}
        \item \dq{It is not easy to determine $\rho$ from this equation. However, one can often recognize certain combinations of differentials that can be made exact by 'ingenious devices'}{696}
    \end{itemize}
    \item For example, consider $x\dd{y}-y\dd{x}=xy^2\dd{x}$.
    \begin{itemize}
        \item We may solve this differential equation by recognizing that it can be rewritten as
        \begin{equation*}
            -x\dd{x} = \frac{y\dd{x}-x\dd{y}}{y^2} = \dd{(x/y)}
        \end{equation*}
        \item Alternatively, we can multiply by the integrating factor $1/y^2$.
    \end{itemize}
\end{itemize}




\end{document}