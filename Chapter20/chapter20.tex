\documentclass[../main.tex]{subfiles}

\pagestyle{main}
\renewcommand{\chaptermark}[1]{\markboth{\chaptername\ \thechapter:\ #1}{}}
\setcounter{chapter}{19}

\begin{document}




\chapter{Differential Equations}\label{cht:20}
\section{Introduction}
\begin{itemize}
    \item \marginnote{9/9:}\textbf{Differential equation}: An equation that involves one or more derivatives, or differentials.
    \item \textbf{Type} (of a differential equation): A differential equation is either an \textbf{ordinary differential equation} or a \textbf{partial differential equation}.
    \item \textbf{Order} (of a differential equation): The order of the highest-order derivative that occurs in the equation.
    \item \textbf{Degree} (of a differential equation): The exponent of the highest power of the highest-order derivative, after the equation has been cleared of fractions and radicals in the dependent variable and its derivatives.
    \item \textbf{Ordinary differential equation}: A differential equation where the only derivatives that appear are those of a dependent variable $y$ varying as a function of a single independent variable $x$.
    \item \textbf{Partial differential equation}: A differential equation where partial derivatives appear.
    \item \dq{For example,
    \begin{equation*}
        \left( \dv[3]{y}{x} \right)^2+\left( \dv[2]{y}{x} \right)^5+\frac{y}{x^2+1} = \e[x]
    \end{equation*}
    is an ordinary differential equation, of order three and degree two}{691}
    \item \textcite{bib:Thomas} does not include a systematic treatment of partial differential equations, so he recommends Chapter 10 of \textcite{bib:Kaplan}.
    \item If $A,B,C$ are radioactive substances such that $A$ decomposes into $B$ at a rate proportional to the amount of $A$ present (proportionality constant $k_1$), $B$ decomposes into $C$ at a rate proportional to the amount of $A$ present (proportionality constant $k_2$), and $C$ decomposes into $A$ at a rate proportional to the amount of $C$ present (proportionality constant $k_3$), we can write the following system of differential equations.
    \begin{align*}
        \dv{x}{t} &= -k_1x+k_3z&
        \dv{y}{t} &= k_1x-k_2y&
        \dv{z}{t} &= k_2y-k_3z
    \end{align*}
    \begin{itemize}
        \item Since $\dv*{x}{t}+\dv*{y}{t}+\dv*{z}{t}=0$ in this case, our solution is that $x+y+z=C$, i.e., the statement that if the amounts of substances change in this manner, the total amount of substance present remains constant.
    \end{itemize}
\end{itemize}



\section{Solutions}
\begin{itemize}
    \item \textbf{Solution} (of a differential equation): A function $y=f(x)$ such that the differential equation is identically satisfied when $y$ and its derivatives are replaced throughout by $f(x)$ and its corresponding derivatives.
    \item Differential equations often have solutions in which certain arbitrary constants occur.
    \begin{itemize}
        \item However, these constants can often be resolved into specific values with the help of initial conditions.
        \item In fact, \dq{a differential equation of order $n$ will in general possess a solution involving $n$ arbitrary constants}{693}
        \begin{itemize}
            \item There is a more precise theorem that implies this result, but \textcite{bib:Thomas} neither states nor proves it.
        \end{itemize}
    \end{itemize}
    \item \textbf{General solution}: A solution that still contains all arbitrary constants arising from the solving process.
    \item Since finding general solutions requires calculus and finding specific solutions from general solutions and initial conditions only requires algebra, we will focus on finding general solutions.
    \item Note that this is only an introduction; for a more exhaustive treatment of differential equations, refer to \textcite{bib:MartinReissner}.
\end{itemize}



\section{First-Order Equations with Variables Separable}
\begin{itemize}
    \item If it is possible to collect all $y$-terms with $\dd{y}$ and all $x$-terms with $\dd{x}$, i.e., if it is possible to write the equation in the form
    \begin{equation*}
        f(y)\dd{y}+g(x)\dd{x} = 0
    \end{equation*}
    then the general solution is
    \begin{equation*}
        \int f(y)\dd{y}+\int g(x)\dd{x} = C
    \end{equation*}
    where $C$ is an arbitrary constant.
\end{itemize}



\section{First-Order Homogeneous Equations}
\begin{itemize}
    \item \textbf{Homogeneous} (differential equation): A differential equation that can be put into the form
    \begin{equation*}
        \dv{y}{x} = F(y/x)
    \end{equation*}
    where $F(y/x)$ is some function of $y/x$.
    \item Solving a first-order homogeneous differential equation.
    \begin{itemize}
        \item We use $u$-substitution.
        \item If we let $u=y/x$, then $\dv*{y}{x}=u+x\dv*{u}{x}$ by the product rule.
        \item Thus, the differential equation can be rewritten in the form $u+x\dv*{u}{x}=F(u)$, which can be solved in terms of $u$ and $x$ via separation of variables as follows.
        \begin{equation*}
            \frac{\dd{x}}{x}+\frac{\dd{u}}{u-F(u)} = 0
        \end{equation*}
        \item Once the above is solved, we can return the substitution to obtain our final solution.
    \end{itemize}
    \item \dq{Show that the equation $(x^2+y^2)\dd{x}+2xy\dd{y}=0$ is homogeneous, and solve it}{694}
    \begin{itemize}
        \item Via basic algebra, we can rewrite the above in the form
        \begin{align*}
            \dv{y}{x} &= -\frac{x^2+y^2}{2xy}\\
            &= -\frac{1+(y/x)^2}{2(y/x)}
        \end{align*}
        implying that it is homogeneous with $F(u)=-(1+u^2)/(2u)$.
        \item Therefore, the only remaining task is to solve
        \begin{equation*}
            \frac{\dd{x}}{x}+\frac{\dd{u}}{u-[-(1+u^2)/(2u)]} = 0
        \end{equation*}
        via separation of variables integration.
        \item After doing so, we obtain
        \begin{align*}
            \ln|x|+\frac{1}{3}\ln(1+3u^2) &= \frac{1}{3}\ln C\\
            x^3(1+3u^2) &= \pm C
        \end{align*}
        \item Therefore, returning our substitution, we have that
        \begin{equation*}
            x(x^2+3y^2) = C
        \end{equation*}
        is our solution.
    \end{itemize}
\end{itemize}



\section{First-Order Linear Equations}
\begin{itemize}
    \item \textbf{Degree} (of a term in a differential equation): The sum of the exponents of the dependent variable and any of its derivatives in a given term.
    \begin{itemize}
        \item For example, $(\dv*[2]{y}{x})$ is of degree one but $y(\dv*{y}{x})$ is of degree two.
    \end{itemize}
    \item \textbf{Linear differential equation}: A differential equation such that every term is of degree zero or degree one.
    \begin{equation*}
        \dv{y}{x}+Py = Q
    \end{equation*}
    \begin{itemize}
        \item A linear differential equation of first order can always be put into the above form, where $P,Q$ are functions of $x$.
    \end{itemize}
    \item \textbf{Integrating factor}: A function $\rho$ of the independent variable $x$ such that if the differential equation at hand is multiplied by $\rho$, it will compress into a form that is easier to integrate.
    \begin{itemize}
        \item For first-order linear differential equations, the function $\rho$ that we seek makes the left-hand side becomes the derivative of the product $\rho y$.
    \end{itemize}
    \item Deriving $\rho$ in terms of the values given in a general first-order linear differential equation.
    \begin{itemize}
        \item Multiplying by $\rho$, we have
        \begin{equation*}
            \rho\dv{y}{x}+\rho Py = \rho Q
        \end{equation*}
        \item Thus, since we want
        \begin{equation*}
            \rho\dv{y}{x}+\rho Py = \dv{x}(\rho y) = \rho\dv{y}{x}+\dv{\rho}{x}y
        \end{equation*}
        we must have by comparing terms that
        \begin{equation*}
            \dv{\rho}{x} = \rho P
        \end{equation*}
        \item It follows by separation of variables integration that
        \begin{align*}
            \frac{\dd{\rho}}{\rho} &= P\dd{x}\\
            \ln|\rho| &= \int P\dd{x}+\ln C\\
            \rho &= \pm C\e[\int P\dd{x}]\\
            \rho &= \e[\int P\dd{x}]
        \end{align*}
        \begin{itemize}
            \item Note that we can choose $\pm C=1$ since in the equation $\rho\dv*{y}{x}+\rho Py=\rho Q$, any $C$ term can be divided out of both sides anyways.
        \end{itemize}
    \end{itemize}
    \item With the help of the integrating factor, we can now derive the general solution to a first-order linear differential equation as follows.
    \begin{align*}
        \dv{y}{x}+Py &= Q\\
        \rho\dv{y}{x}+\rho Py &= \rho Q\\
        \dv{x}(\rho y) &= \rho Q\\
        \rho y &= \int\rho Q\dd{x}+C\\
        y &= \frac{1}{\e[\int P\dd{x}]}\left( \int\e[\int P\dd{x}]Q\dd{x}+C \right)
    \end{align*}
    \item Note that a first-order linear differential equation may also be separable, or homogeneous. In such cases, we have a choice of solution methods.
\end{itemize}



\section{First-Order Equations With Exact Differentials}
\begin{itemize}
    \item Refer to Section 15.13 for the method of solving exact differentials.
    \item Every first-order differential equation $P(x,y)\dd{x}+Q(x,y)\dd{y}=0$ can be made exact by multiplication by an integrating factor $\rho(x,y)$ having the property that
    \begin{equation*}
        \pdv{y}[\rho(x,y)P(x,y)] = \pdv{x}[\rho(x,y)Q(x,y)]
    \end{equation*}
    \begin{itemize}
        \item \dq{It is not easy to determine $\rho$ from this equation. However, one can often recognize certain combinations of differentials that can be made exact by 'ingenious devices'}{696}
    \end{itemize}
    \item For example, consider $x\dd{y}-y\dd{x}=xy^2\dd{x}$.
    \begin{itemize}
        \item We may solve this differential equation by recognizing that it can be rewritten as
        \begin{equation*}
            -x\dd{x} = \frac{y\dd{x}-x\dd{y}}{y^2} = \dd{(x/y)}
        \end{equation*}
        \item Alternatively, we can multiply by the integrating factor $1/y^2$.
    \end{itemize}
\end{itemize}



\section{Special Types of Second-Order Equations}
\begin{itemize}
    \item \marginnote{9/10:}\dq{Certain types of second-order differential equations, of which the general form is
    \begin{equation*}
        F\left( x,y,\dv{y}{x},\dv[2]{y}{x} \right) = 0
    \end{equation*}
    can be reduced to first-order equations by a suitable change of variables}{697}
    \item Equations with the dependent variable missing:
    \begin{equation*}
        F\left( x,\dv{y}{x},\dv[2]{y}{x} \right) = 0
    \end{equation*}
    \begin{itemize}
        \item Equations of the above form can be reduced to a first-order equation by substituting $p=\dv*{y}{x}$ and $\dv*{p}{x}=\dv*[2]{y}{x}$.
        \item Then if we can solve for $p(x,C_1)$, we can integrate again to solve for $y$.
    \end{itemize}
    \item Equations with the independent variable missing:
    \begin{equation*}
        F\left( y,\dv{y}{x},\dv[2]{y}{x} \right) = 0
    \end{equation*}
    \begin{itemize}
        \item Equations of the above form can be reduced to a first-order equation by substituting $p=\dv*{y}{x}$ and $\dv*[2]{y}{x}=p\dv*{p}{x}$.
        \item Then if we solve for $p(y,C_1)$, we can further solve for $y$.
    \end{itemize}
    \item For example, consider
    \begin{equation*}
        \dv[2]{y}{x}+y = 0
    \end{equation*}
    \begin{itemize}
        \item By the second method above, we get
        \begin{align*}
            0 &= p\dv{p}{y}+y\\
            &= p\dd{p}+y\dd{y}\\
            \frac{C_1^2}{2} &= \frac{p^2}{2}+\frac{y^2}{2}
        \end{align*}
        \item Returning the substitutions, we have
        \begin{align*}
            \dv{y}{x} &= \pm\sqrt{C_1^2-y^2}\\
            \frac{\dd{y}}{\sqrt{C_1^2-y^2}} &= \pm\dd{x}\\
            \sin^{-1}\left( \frac{y}{C_1} \right) &= \pm(x+C_2)\\
            y &= \pm C_1\sin(x+C_2)
        \end{align*}
    \end{itemize}
\end{itemize}



\section{Linear Equations With Constant Coefficients}
\begin{itemize}
    \item \textbf{Linear differential equation of order $\bm{n}$}: An equation of the form
    \begin{equation*}
        \dv[n]{y}{x}+a_1\dv[n-1]{y}{x}+a_2\dv[n-2]{y}{x}+\cdots+a_{n-1}\dv{y}{x}+a_ny = F(x)
    \end{equation*}
    which is linear in $y$ and its derivatives, where $a_1,\dots,a_n$ are functions of $x$.
    \item \textbf{Homogeneous} (linear differential equation of order $n$): A linear differential equation of order $n$ such that $F(x)=0$.
    \item \textbf{Nonhomogeneous} (linear differential equation of order $n$): A linear differential equation of order $n$ that is not homogeneous, i.e., such that $F(x)\neq 0$.
    \item Let $Df(x)=(\dv*{x})f(x)$. We similarly denote higher order derivatives with $D^nf(x)=(\dv*[n]{x})f(x)$.
    \item \textbf{Linear differential operator}: A polynomial in $D$, meant to be interpreted as an operator which, when applied to $f(x)$, produces a linear combination of $f$ and its successive derivatives. \emph{Denoted by} $\bm{L}$.
    \begin{itemize}
        \item For example, $(D^2+D-2)f(x)=\dv*[2]{f(x)}{x}+\dv*{f(x)}{x}-2f(x)$.
    \end{itemize}
    \item Linear differential operators are additive and multiplicative:
    \begin{align*}
        (L_1+L_2)f(x) &= L_1f(x)+L_2f(x)&
        L_1L_2f(x) &= L_1(L_2f(x))
    \end{align*}
    \begin{itemize}
        \item Thus, they satisfy basic algebraic laws, so we may express $D^2-D-2$ as $(D+2)(D-2)$ for example, making the differential equation easier to solve.
    \end{itemize}
\end{itemize}



\section{Homogeneous Linear Second-Order Differential Equations With Constant Coefficients}
\begin{itemize}
    \item \textbf{Characteristic equation} (of a homogeneous linear differential equation with constant coefficients): The polynomial obtained by substituting $r^n$ for each $\dv*[n]{y}{x}$ term. Note that we substitute $1=r^0=\dv*[0]{y}{x}=y$ for $y$, as well.
    \item Suppose we wish to solve an equation of the form
    \begin{equation*}
        \dv[2]{y}{x}+2a\dv{y}{x}+by = 0
    \end{equation*}
    \begin{itemize}
        \item Then in the notation of the last section, $(D^2+2aD+b)y=(D-r_1)(D-r_2)y=0$, where $r_1,r_2$ are the roots of the polynomial $r^2+2ar+b$.
        \item Thus, we can solve this differential equation by solving the sub-equations
        \begin{align*}
            (D-r_2)y &= u&
            (D-r_1)u &= 0
        \end{align*}
        \item It follows that $u=C_1\e[r_1x]$, meaning that to solve
        \begin{equation*}
            \dv{y}{x}-r_2y = C_1\e[r_1x]
        \end{equation*}
        as a first-order homogeneous equation with integrating factor $\rho=\e[-r_2x]$, we need to divide into two cases ($r_1\neq r_2$ and $r_1=r_2$).
        \begin{itemize}
            \item If $r_1\neq r_2$, then
            \begin{equation*}
                y = C_1\e[r_1x]+C_2\e[r_2x]
            \end{equation*}
            \item If $r_1=r_2$, then
            \begin{equation*}
                y = (C_1x+C_2)\e[r_2x]
            \end{equation*}
        \end{itemize}
    \end{itemize}
    \item Imaginary roots.
    \begin{itemize}
        \item Suppose the polynomial has roots $r_1=\alpha+i\beta$ and $r_2=\alpha-i\beta$.
        \item If $\beta=0$, then the roots are real and we just use the $r_1=r_2$ formula above.
        \item However, if $\beta\neq 0$, then the roots are complex and we can modify the $r_1\neq r_2$ solution above into a more convenient form.
        \begin{align*}
            y &= c_1\e[(\alpha+i\beta)x]+c_2\e[(\alpha-i\beta)x]\\
            &= \e[\alpha x][c_1\e[i\beta x]+c_2\e[-i\beta x]]\\
            &= \e[\alpha x][c_1(\cos\beta x+i\sin\beta x)+c_2(\cos\beta x-i\sin\beta x)]\\
            &= \e[\alpha x][(c_1+c_2)\cos\beta x+i(c_1-c_2)\sin\beta x]\\
            &= \e[\alpha x][C_1\cos\beta x+C_2\sin\beta x]
        \end{align*}
        \begin{itemize}
            \item Note that $C_1$ and $C_2$ will generally be real.
        \end{itemize}
    \end{itemize}
\end{itemize}



\section{Nonhomogeneous Linear Second-Order Differential Equations With Constant Coefficients}
\begin{itemize}
    \item Suppose we wish to solve an equation of the form
    \begin{equation*}
        \dv[2]{y}{x}+2a\dv{y}{x}+by = F(x)
    \end{equation*}
    \begin{itemize}
        \item First, find the general solution $y_h$ to the homogeneous form of the above equation, $\dv*[2]{y}{x}+2a\dv*{y}{x}+by=0$, i.e.,
        \begin{equation*}
            y_h = C_1u_1(x)+C_2u_2(x)
        \end{equation*}
        where $u_1,u_2$ equal the appropriate corresponding functions, as derived in Section 20.9.
        \item There are two possibilities from here.
        \item Inspection.
        \begin{itemize}
            \item We can find by inspection one particular function $y=y_p(x)$ which satisfies the original differential equation, in which case
            \begin{equation*}
                y = y_h(x)+y_p(x)
            \end{equation*}
        \end{itemize}
        \item Variation of parameters.
        \begin{itemize}
            \item Replace $C_1,C_2$ with $v_1,v_2$, which we let be functions of $x$.
            \item The goal is now just to impose conditions that allow us to solve for $v_1,v_2$.
            \item As a first condition, require that
            \begin{equation*}
                v_1'u_1+v_2'u_2 = 0
            \end{equation*}
            \item Thus, using the substitutions of $v_1,v_2$ and the above condition, we have that
            \begin{align*}
                y &= v_1u_1+v_2u_2&
                \dv{y}{x} &= v_1u_1'+v_1'u_1+v_2u_2'+v_2'u_2&
                \dv[2]{y}{x} &= v_1u_1''+v_1'u_1'+v_2u_2''+v_2'u_2'\\
                &&
                &= v_1u_1'+v_2u_2'
            \end{align*}
            \item It follows by substitution into the original equation that
            \begin{align*}
                F(x) &= (v_1u_1''+v_1'u_1'+v_2u_2''+v_2'u_2')+2a(v_1u_1'+v_2u_2')+b(v_1u_1+v_2u_2)\\
                &= v_1\left( \dv[2]{u_1}{x}+2a\dv{u_1}{x}+bu_1 \right)+v_2\left( \dv[2]{u_2}{x}+2a\dv{u_2}{x}+bu_2 \right)+v_1'u_1'+v_2'u_2'
            \end{align*}
            \item But since $u_1,u_2$ are solutions to the homogeneous form of the original equation, the terms in the parentheses above equal zero.
            \item Thus, as a second condition, we require that
            \begin{equation*}
                v_1'u_1'+v_2'u_2' = F(x)
            \end{equation*}
            \item In summary, to solve for $v_1,v_2$ after making the appropriate substitutions, solve
            \begin{align*}
                v_1'u_1+v_2'u_2 &= 0&
                v_1'u_1'+v_2'u_2' &= F(x)
            \end{align*}
            as a two-variable system of equations in $v_1',v_2'$. Then recover $v_1,v_2$ with integration.
        \end{itemize}
    \end{itemize}
    \item For example, consider the following equation.
    \begin{equation*}
        \dv[2]{y}{x}+2\dv{y}{x}-3y = 6
    \end{equation*}
    \begin{itemize}
        \item The characteristic equation of the homogeneous form is $r^2+2r-3=(r-1)(r+3)=0$.
        \item Thus, $y_h=C_1\e[-3x]+C_2\e[x]$.
        \item Substitute $u_1(x)=\e[-3x]$ and $u_2(x)=\e[x]$.
        \item Consequently, our two-variable system of equations is
        \begin{align*}
            v_1'\e[-3x]+v_2'\e[x] &= 0&
            v_1'(-3\e[-3x])+v_2'\e[x] &= 6
        \end{align*}
        \item Solving the above yields
        \begin{align*}
            v_1' &= -\frac{3}{2}\e[3x]&
            v_2' &= \frac{3}{2}\e[-x]
        \end{align*}
        \item It follows that
        \begin{align*}
            v_1 &= \int-\frac{3}{2}\e[3x]\dd{x} = -\frac{1}{2}\e[3x]+c_1&
            v_2 &= \int\frac{3}{2}\e[-x]\dd{x} = -\frac{3}{2}\e[-x]+c_2
        \end{align*}
        \item Therefore, we have that
        \begin{align*}
            y &= v_1u_1+v_2u_2\\
            &= \left( -\frac{1}{2}\e[3x]+c_1 \right)\e[-3x]+\left( -\frac{3}{2}\e[-x]+c_2 \right)\e[x]\\
            &= -2+c_1\e[-3x]+c_2\e[x]
        \end{align*}
    \end{itemize}
\end{itemize}



\section{Higher-Order Linear Differential Equations With Constant Coefficients}
\begin{itemize}
    \item The methods of Sections 20.9-20.10 extend to equations of higher order.
    \item If the roots $r_1,\dots,r_n$ of the characteristic polynomial are all distinct, then the solution of the homogenous equation is
    \begin{equation*}
        y_h = c_1\e[r_1x]+\cdots+c_n\e[r_nx]
    \end{equation*}
    \item If the roots of the characteristic polynomial are not all distinct, then the portion of $y_h$ corresponding to a root $r$ of multiplicity $m$ is to be replaced by
    \begin{equation*}
        (C_1x^{m-1}+C_2x^{m-2}+\cdots+C_m)\e[rx]
    \end{equation*}
    \item If the general solution of the homogeneous equation is $y_h=C_1u_1+\cdots+C_nu_n$, then $y=v_1u_1+\cdots+v_nu_n$ will be a solution of the nonhomogeneous equation if and only if
    \begin{align*}
        v_1'u_1+\cdots+v_n'u_n &= 0\\
        v_1'\dv{u_1}{x}+\cdots+v_n'\dv{u_n}{x} &= 0\\
        % v_1'\dv[2]{u_1}{x}+\cdots+v_n'\dv[2]{u_n}{x} &= 0\\
        &\hspace{5.5pt}\vdots\\
        v_1'\dv[n-2]{u_1}{x}+\cdots+v_n'\dv[n-2]{u_n}{x} &= 0\\
        v_1'\dv[n-1]{u_1}{x}+\cdots+v_n'\dv[n-1]{u_n}{x} &= F(x)
    \end{align*}
    \begin{itemize}
        \item These equations can be solved for $v_1',\dots,v_n'$ and then integrated.
    \end{itemize}
\end{itemize}



\section{Vibrations}
\begin{itemize}
    \item \textcite{bib:Thomas} uses differential equations and Newton's Laws to derive the undamped and damped motion of a system vibrating under a linear restoring force.
\end{itemize}




\end{document}