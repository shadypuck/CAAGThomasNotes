\documentclass[../main.tex]{subfiles}

\pagestyle{main}
\renewcommand{\chaptermark}[1]{\markboth{\chaptername\ \thechapter:\ #1}{}}
\setcounter{chapter}{16}

\begin{document}




\chapter{Vector Analysis}\label{cht:17}
\section{Introduction: Vector Fields}
\begin{itemize}
    \item \marginnote{12/29:}In this chapter, we will consider vector functions of several variables, such as the function giving the velocity $\vb{v}=\vb{F}(x,y,z,t)$ of a particle in a fluid located at position $(x,y,z)$ at time $t$.
    \item \textbf{Steady-state flow}: A flow for which the velocity function does not depend on the time $t$.
    \item \textbf{Vector field}: The collection of all vectors $\vb{F}(P)$ assigned to each point $P$ in a region $G$.
    \item \textbf{Gradient field}: The vector field defined for points in the domain $G$ of a scalar function $T$ such that $\vb{F}(P)=\nabla T(P)$.
\end{itemize}



\section{Surface Integrals}
\begin{itemize}
    \item \marginnote{12/30:}Just like we have $\dd{s}=\sqrt{1+f_x^2}\dd{x}$, we have
    \begin{equation*}
        \dd{\sigma} = g(x,y)\dd{A}
    \end{equation*}
    where $\dd{\sigma}$ is "an element of surface area in the tangent plane that approximates the corresponding portion $\Delta\sigma$ of the surface itself" \parencite[581]{bib:Thomas} and $g(x,y)=\sqrt{1+f_x^2+f_y^2}$.
    \item Thus, we can think of surface area as either the lefthand or righthand side of the below equation.
    \begin{equation*}
        \iint\limits_\Sigma\dd{\sigma} = \iint\limits_Rg(x,y)\dd{A}
    \end{equation*}
    \begin{itemize}
        \item The lefthand interpretation sums infinitely many, infinitely small pieces $\dd{\sigma}$ of the surface $\Sigma$.
        \item The righthand interpretation sums infinitely many, infinitely small pieces $\dd{A}$ of the shadow $R$ of the surface $\Sigma$ on the $xy$-plane, adjusted by the factor $g(x,y)$.
    \end{itemize}
    \item These formulations are important because sometimes we want to conceive and evaluate an integral of the form $\iint_\Sigma h(x,y,z)\dd{\sigma}$.
    \item \textbf{Surface integral} (of $h(x,y,z)$ over the surface $\Sigma$): The limit as $\Delta\sigma\to 0$ of the sum of every $\Delta\sigma_k$ (composing $\Sigma$) times $h(x,y,z)$ for some $(x,y,z)\in\Delta\sigma_k$. Mathematically,
    \begin{equation*}
        \iint\limits_{\Sigma} h(x,y,z)\dd{\sigma} = \lim_{\Delta\sigma\to 0}\sum_{k=1}^nh(x_k,y_k,z_k)\, \Delta\sigma_k
    \end{equation*}
    \begin{itemize}
        \item Consider a surface $\Sigma$ consisting of all points $P(x,y,z)$ satisfying $z=f(x,y)$ for $(x,y)\in R$, where $R$ is a closed, bounded region of the $xy$-plane and $f,f_x,f_y$ are continuous throughout $R$ and its boundary.
        \item Approximate $R$ by dividing it into $n$ rectangles using lines parallel to the $y$-axis spaced $\Delta x$ apart and lines parallel to the $x$-axis spaced $\Delta y$ apart.
        \item Let the part of $\Sigma$ above each rectangle be denoted by $\Delta\sigma_k$ for some $1\leq k\leq n$.
        \item Now if $P_k(x_k,y_k,z_k)$ is a point in $\Delta\sigma_k$, we can consider the above sum and take its limit.
    \end{itemize}
    \item \marginnote{12/31:}To evaluate the surface integral, we substitute $\Delta\sigma_k=g(x_k,y_k)\, \Delta x\, \Delta y$ and $z_k=f(x_k,y_k)$ in the sum, and take iterated integrals over $R$ (the shadow of $\Sigma$ on the $xy$-plane) instead of $\Sigma$.
    \begin{equation*}
        \iint\limits_\Sigma h(x,y,z)\dd{\sigma} = \iint\limits_R h[x,y,f(x,y)]g(x,y)\dd{x}\dd{y}
    \end{equation*}
    \item We now explore a useful surface integration technique through a problem.
    \item Evaluate $\iint(x^2+y^2)\dd{\sigma}$ over the hemisphere $\Sigma$ described by $z=\sqrt{a^2-x^2-y^2}$.
    \begin{itemize}
        \item Because of a \emph{sphere} $2\Sigma$ of radius $a$'s high degree of symmetry,
        \begin{equation*}
            \iint\limits_{2\Sigma}x^2\dd{\sigma} = \iint\limits_{2\Sigma}y^2\dd{\sigma}
            = \iint\limits_{2\Sigma}z^2\dd{\sigma}
            = \frac{1}{3}\iint\limits_{2\Sigma}(x^2+y^2+z^2)\dd{\sigma}
            = \frac{1}{3}\iint\limits_{2\Sigma}a^2\dd{\sigma}
        \end{equation*}
        Thus, for the \emph{hemisphere} $\Sigma$,
        \begin{align*}
            \iint\limits_\Sigma(x^2+y^2)\dd{\sigma} &= \frac{1}{2}\iint\limits_{2\Sigma}(x^2+y^2)\dd{\sigma}\\
            &= \frac{1}{2}\left( \iint\limits_{2\Sigma}x^2\dd{\sigma}+\iint\limits_{2\Sigma}y^2\dd{\sigma} \right)\\
            &= \frac{1}{2}\left( \frac{1}{3}\iint\limits_{2\Sigma}a^2\dd{\sigma}+\frac{1}{3}\iint\limits_{2\Sigma}a^2\dd{\sigma} \right)\\
            &= \frac{a^2}{3}\iint\limits_{2\Sigma}\dd{\sigma}\\
            &= \frac{a^2}{3}\cdot 4\pi a^2\\
            &= \frac{4}{3}\pi a^4
        \end{align*}
    \end{itemize}
    \item Alternate formulations of $\dd{\sigma}$.
    \begin{itemize}
        \item Let the surface $\Sigma$ be defined by the equation $F(x,y,z)=0$.
        \item For the same reasons discussed in Chapter \ref{cht:17},
        \begin{equation*}
            \dd{\sigma} = \frac{\dd{A}}{\cos\phi}
        \end{equation*}
        where $\phi$ is the angle between $\vb{N}=\nabla F$ and the unit vector normal to the plane onto which $\Sigma$ is projected, which we will take to be the $xy$-plane at first (this means that this normal vector is $\vb{k}$).
        \item Since
        \begin{equation*}
            \cos\phi = \frac{\vb{N}\cdot\vb{k}}{|\vb{N}|\, |\vb{k}|} = \frac{|F_z|}{\sqrt{F_x^2+F_y^2+F_z^2}}
        \end{equation*}
        we thus have that
        \begin{equation*}
            \dd{\sigma} = \frac{\sqrt{F_x^2+F_y^2+F_z^2}}{|F_z|}\dd{x}\dd{y}
        \end{equation*}
        \item Note that if we project $\Sigma$ onto a different plane, an analog to the above can easily be derived.
    \end{itemize}
\end{itemize}



\section{Line Integrals}
\begin{itemize}
    \item \textbf{Line integral} (of $w(x,y,z)$ along the curve $C$ from $A$ to $B$): The limit as $\Delta s\to 0$ of the sum of every $\Delta s_k$ (composing the section of $C$ between points $A$ and $B$ along $C$) times $w(x,y,z)$ for some $(x,y,z)\in\Delta s_k$. Mathematically,
    \begin{equation*}
        \int_Cw\dd{s} = \lim_{\Delta s\to 0}\sum_{k=1}^n w(x_k,y_k,z_k)\, \Delta s_k
    \end{equation*}
    \begin{itemize}
        \item Suppose that $C$ is a directed curve in three-space from $A$ to $B$. Let $w(x,y,z)$ be a scalar function of position that is continuous in a region $D$ containing $C$.
        \item Divide $C$ into $n$ segments, and let $P_k(x_k,y_k,z_k)$ be an arbitrary point on the $k$th subarc.
        \item If the above sum has a limit as $n\to\infty$ and the largest $\Delta s_k\to 0$, and if this limit is the same for all ways of subdividing $C$ and all choices of the points $P_k$, then we call this limit the line integral.
    \end{itemize}
    \item If $C$ is parameterized by the functions $x=f(t)$, $y=g(t)$, and $z=h(t)$ for $t_A\leq t\leq t_B$, where $f,g,h$ are continuous and have bounded and piecewise-continuous first derivatives on $[t_A,t_B]$, then we may evaluate the line integral of $w(x,y,z)$ along $C$ from $A$ to $B$ with the following formula.
    \begin{equation*}
        \int_Cw\dd{s} = \int_{t_A}^{t_B}w[f(t),g(t),h(t)]\sqrt{\left( \dv{f}{t} \right)^2+\left( \dv{g}{t} \right)^2+\left( \dv{h}{t} \right)^2}\dd{t}
    \end{equation*}
    \begin{itemize}
        \item Note that the line integral is the same for any appropriate parameterization of $C$, or no parameterization.
    \end{itemize}
\end{itemize}




\end{document}