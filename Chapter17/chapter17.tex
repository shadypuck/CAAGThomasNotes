\documentclass[../main.tex]{subfiles}

\pagestyle{main}
\renewcommand{\chaptermark}[1]{\markboth{\chaptername\ \thechapter:\ #1}{}}
\setcounter{chapter}{16}

\begin{document}




\chapter{Vector Analysis}\label{cht:17}
\section{Introduction: Vector Fields}
\begin{itemize}
    \item \marginnote{12/29:}In this chapter, we will consider vector functions of several variables, such as the function giving the velocity $\vb{v}=\vb{F}(x,y,z,t)$ of a particle in a fluid located at position $(x,y,z)$ at time $t$.
    \item \textbf{Steady-state flow}: A flow for which the velocity function does not depend on the time $t$.
    \item \textbf{Vector field}: The collection of all vectors $\vb{F}(P)$ assigned to each point $P$ in a region $G$.
    \item \textbf{Gradient field}: The vector field defined for points in the domain $G$ of a scalar function $T$ such that $\vb{F}(P)=\nabla T(P)$.
\end{itemize}



\section{Surface Integrals}
\begin{itemize}
    \item \marginnote{12/30:}Just like we have $\dd{s}=\sqrt{1+f_x^2}\dd{x}$, we have
    \begin{equation*}
        \dd{\sigma} = g(x,y)\dd{A}
    \end{equation*}
    where $\dd{\sigma}$ is "an element of surface area in the tangent plane that approximates the corresponding portion $\Delta\sigma$ of the surface itself" \parencite[581]{bib:Thomas} and $g(x,y)=\sqrt{1+f_x^2+f_y^2}$.
    \item Thus, we can think of surface area as either the lefthand or righthand side of the below equation.
    \begin{equation*}
        \iint\limits_\Sigma\dd{\sigma} = \iint\limits_Rg(x,y)\dd{A}
    \end{equation*}
    \begin{itemize}
        \item The lefthand interpretation sums infinitely many, infinitely small pieces $\dd{\sigma}$ of the surface $\Sigma$.
        \item The righthand interpretation sums infinitely many, infinitely small pieces $\dd{A}$ of the shadow $R$ of the surface $\Sigma$ on the $xy$-plane, adjusted by the factor $g(x,y)$.
    \end{itemize}
    \item These formulations are important because sometimes we want to conceive and evaluate an integral of the form $\iint_\Sigma h(x,y,z)\dd{\sigma}$.
    \item \textbf{Surface integral} (of $h(x,y,z)$ over the surface $\Sigma$): The limit as $\Delta\sigma\to 0$ of the sum of every $\Delta\sigma_k$ (composing $\Sigma$) times $h(x,y,z)$ for some $(x,y,z)\in\Delta\sigma_k$. Mathematically,
    \begin{equation*}
        \iint\limits_{\Sigma} h(x,y,z)\dd{\sigma} = \lim_{\Delta\sigma\to 0}\sum_{k=1}^nh(x_k,y_k,z_k)\, \Delta\sigma_k
    \end{equation*}
    \begin{itemize}
        \item Consider a surface $\Sigma$ consisting of all points $P(x,y,z)$ satisfying $z=f(x,y)$ for $(x,y)\in R$, where $R$ is a closed, bounded region of the $xy$-plane and $f,f_x,f_y$ are continuous throughout $R$ and its boundary.
        \item Approximate $R$ by dividing it into $n$ rectangles using lines parallel to the $y$-axis spaced $\Delta x$ apart and lines parallel to the $x$-axis spaced $\Delta y$ apart.
        \item Let the part of $\Sigma$ above each rectangle be denoted by $\Delta\sigma_k$ for some $1\leq k\leq n$.
        \item Now if $P_k(x_k,y_k,z_k)$ is a point in $\Delta\sigma_k$, we can consider the above sum and take its limit.
    \end{itemize}
\end{itemize}




\end{document}