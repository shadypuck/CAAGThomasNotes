\documentclass[../main.tex]{subfiles}

\pagestyle{main}
\renewcommand{\chaptermark}[1]{\markboth{\chaptername\ \thechapter:\ #1}{}}
\setcounter{chapter}{16}

\begin{document}




\chapter{Vector Analysis}\label{cht:17}
\section{Introduction: Vector Fields}
\begin{itemize}
    \item \marginnote{12/29:}In this chapter, we will consider vector functions of several variables, such as the function giving the velocity $\vb{v}=\vb{F}(x,y,z,t)$ of a particle in a fluid located at position $(x,y,z)$ at time $t$.
    \item \textbf{Steady-state flow}: A flow for which the velocity function does not depend on the time $t$.
    \item \textbf{Vector field}: The collection of all vectors $\vb{F}(P)$ assigned to each point $P$ in a region $G$.
    \item \textbf{Gradient field}: The vector field defined for points in the domain $G$ of a scalar function $T$ such that $\vb{F}(P)=\nabla T(P)$.
\end{itemize}




\end{document}