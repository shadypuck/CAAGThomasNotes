\documentclass[../main.tex]{subfiles}

\pagestyle{main}
\renewcommand{\chaptermark}[1]{\markboth{\chaptername\ \thechapter: #1}{}}

\begin{document}




\chapter{The Rate of Change of a Function}
\section{Introduction}
\begin{itemize}
    \item \marginnote{7/3:}Discusses the importance of calculus, when it should be used, and why one should study it.
    \item \textbf{Analytic geometry}: \dq{Uses algebraic methods and equations to study geometric problems. Conversely, it permits us to visualize algebraic equations in terms of geometric curves}{2}
\end{itemize}



\section{Coordinates}
\begin{itemize}
    \item \dq{The basic idea in analytic geometry is the establishment of a one-to-one correspondence between the points of a plane on the one hand and pairs of numbers $(x,y)$ on the other hand}{2}
    \item Such a correspondence is most commonly established as follows.
    \begin{figure}[h!]
        \centering
        \begin{tikzpicture}
            \footnotesize
            \draw [-stealth] (-2.8,0) -- (2.8,0) node[right]{$x$};
            \draw [-stealth] (0,-3) -- (0,3) node[above]{$y$};

            \draw (-2.1,0.1) -- ++(0,-0.2) node[below]{$-a$};
            \draw (1,0.1) -- ++(0,-0.2) node[below]{$+1$};
            \draw (2.1,0.1) -- ++(0,-0.2) node[below]{$+a$};
            \draw (0.1,-1.7) -- ++(-0.2,0) node[left]{$-b$};
            \draw (0.1,0.8) -- ++(-0.2,0) node[left]{$+1$};
            \draw (0.1,1.7) -- ++(-0.2,0) node[left]{$+b$};

            \draw [ylx,thick] (2.1,0) -- ++(0,1.7) node[right,black]{$P(a,b)$} -- ++(-2.1,0);

            \node [below left] {$O$};
            \node at (1.8,2.4) {$I(+,+)$};
            \node at (-1.8,2.4) {$II(-,+)$};
            \node at (-1.8,-2.4) {$III(-,-)$};
            \node at (1.8,-2.4) {$IV(+,-)$};
        \end{tikzpicture}
        \caption{Cartesian coordinates.}
        \label{fig:cartesianPlane}
    \end{figure}
    \begin{itemize}
        \item \dq{A horizontal line in the plane, extending indefinitely to the left and to the right, is chosen as the $x$-axis or axis of \textbf{abscissas}. A reference point $O$ on this line and a unit of length are then chosen. The axis is scaled off in terms of this unit of length in such a way that the number zero is attached to $O$, the number $+a$ is attached to the point which is $a$ units to the right of $O$, and $-a$ is attached to the symmetrically located point to the left of $O$. In this way, a one-to-one correspondence is established between points of the $x$-axis and the set of all \textbf{real numbers}}{2-3}
        \item \dq{Now through $O$ take a second, vertical line in the plane, extending indefinitely up and down. This line becomes the $y$-axis, or axis of \textbf{ordinates}. The unit of length used to represent $+1$ on the $y$-axis need not be the same as the unit of length used to represent $+1$ on the $x$-axis. The $y$-axis is scaled off in terms of the unit of length adopted for it, with the positive number $+b$ attached to the point $b$ units above $O$ and negative number $-b$ attached to the symmetrically located point $b$ units below $O$}{3}
        \item \dq{If a line parallel to the $y$-axis is drawn through the point marked $a$ on the $x$-axis, and a line parallel to the $x$-axis is drawn through the point marked $b$ on the $y$-axis, their point of intersection $P$ is to be labeled $P(a,b)$. Thus, given the pair of real numbers $a$ and $b$, we find one and only one point with abscissa $a$ and ordinate $b$, and this point we denote by $P(a,b)$}{3}
        \item \dq{Conversely, if we start with any point $P$ in the plane, we may draw lines through it parallel to the coordinate axes. If these lines intersect the $x$-axis at $a$ and the $y$-axis at $b$, we then regard the pair of numbers $(a,b)$ as corresponding to the point $P$. We say that the coordinates of $P$ are $(a,b)$}{3}
        \item \dq{The two axes divide the plane into four quadrants, called the first quadrant, second quadrant, and so on, and labeled I, II, III, IV in [Figure \ref{fig:cartesianPlane}]. Points in the first quadrant have both coordinates positive, and in the second quadrant the $x$-coordinate (abscissa) is negative and the $y$-coordinate (ordinate) is positive. The notations $(-,-)$ and $(+,-)$ in quadrants III and IV of [Figure \ref{fig:cartesianPlane}] represent the signs of the coordinates of points in these quadrants}{3}
    \end{itemize}
\end{itemize}



\section{Increments}
\begin{itemize}
    \item \textbf{Increments}: The values $\Delta x = x_2-x_1$ and $\Delta y = y_2-y_1$ concerning a particle, the initial position of which is $P_1(x_1,y_1)$ and the terminal position of which is $P_2(x_2,y_2)$.
    \item If the unit of measurement for both axes is the same, then we may express distances in the plane in terms of this unit using the Pythagorean theorem.
\end{itemize}



\section{Slope of a Straight Line}
\begin{itemize}
    \item Let $L$ be a straight line not parallel to the $y$-axis intersecting distinct points $P_1(x_1,y_1)$ and $P_2(x_2,y_2)$. Then $L$ has a \textbf{rise}, \textbf{run}, and \textbf{slope}.
    \item \textbf{Rise}: The increment $\Delta y$.
    \item \textbf{Run}: The increment $\Delta x$.
    \item \textbf{Slope}: The rate of rise per run $m=\frac{\text{rise}}{\text{run}}=\frac{\Delta y}{\Delta x}=\frac{y_2-y_1}{x_2-x_1}$. \emph{Also known as} \textbf{inclination}.
    \begin{figure}[h!]
        \centering
        \begin{subfigure}[b]{0.3\linewidth}
            \centering
            \begin{tikzpicture}
                \footnotesize
                \coordinate (A) at (0,0);
                
                \draw [-stealth,name path=x axis] (-2,0) -- (1.3,0) node[right]{$x$};
                \draw [-stealth,name path=y axis] (0,-0.4) -- (0,2) node[above]{$y$};

                \draw [ylx,thick,name path=line] (-1.9,-0.2) -- (1.2,1.9);

                \path[name intersections={of=x axis and line,by=B}];
                \path[name intersections={of=y axis and line,by=C}];
                \pic[draw,->,pic text={$\phi$},angle eccentricity=1.3]{angle};

                \node [below left] {$O$};
            \end{tikzpicture}
            \caption{Negative $x$-intercept.}
            \label{fig:angleOfInclinationa}
        \end{subfigure}
        \begin{subfigure}[b]{0.3\linewidth}
            \centering
            \begin{tikzpicture}
                \footnotesize
                \coordinate (A) at (2,0);
                
                \draw [-stealth,name path=x axis] (-0.5,0) -- (2.5,0) node[right]{$x$};
                \draw [-stealth,name path=y axis] (0,-0.4) -- (0,2) node[above]{$y$};

                \draw [ylx,thick,name path=line] (1.9,-0.2) -- (-0.4,1.9);

                \path[name intersections={of=x axis and line,by=B}];
                \path[name intersections={of=y axis and line,by=C}];
                \pic[draw,->,pic text={$\phi$},angle eccentricity=1.3]{angle};

                \node [below left] {$O$};
            \end{tikzpicture}
            \caption{Positive $x$-intercept.}
            \label{fig:angleOfInclinationb}
        \end{subfigure}
        \caption{The slope and the angle of inclination.}
        \label{fig:angleOfInclination}
    \end{figure}
    \begin{itemize}
        \item If we chose different distinct points, the slope would be same because the triangles in the Cartesian plane would be similar.
        \item $\Delta y$ is proportional to $\Delta x$ with $m$ as the proportionality factor.
        \item On interpolation: If we're given the values of a function at $(x_1,y_1)$ and $(x_2,y_2)$, then we may approximate the function by a straight line $L$ passing through those two points and approximate the value $f(x)$ for any $x_1\leq x\leq x_2$.
        \item If the scales on both axes are equal, then the slope of $L$ is equal to the tangent of the \textbf{angle of inclination} that $L$ makes with the positive $x$-axis. That is, $m=\tan\phi$ (see Figure \ref{fig:angleOfInclination}).
    \end{itemize}
    \item \textbf{Parallel} (lines): Two lines with equal inclinations ($m_1=m_2$).
    \item \textbf{Perpendicular} (lines): Two lines with inclinations that differ by $90^\circ$ ($m_1=-\frac{1}{m_2}$).
    \begin{itemize}
        \item Note that we can prove the relation between the slopes using the angles of inclination as follows.
        \begin{align*}
            m_1 &= \tan\phi_1\\
            &= \tan\left( \phi_2+90^\circ \right)\\
            &= -\cot\phi_2\\
            &= -\frac{1}{\tan\phi_2}\\
            &= -\frac{1}{m_2}
        \end{align*}
    \end{itemize}
\end{itemize}



\section{Equations of a Straight Line}
\begin{itemize}
    \item \marginnote{7/5:}How do you know if $P(x,y)$ is a point on the line $P_1P_2$ through distinct points $P_1(x_1,y_1)$ and $P_2(x_2,y_2)$?
    \begin{itemize}
        \item If $x_1=x_2$, then $P_1P_2$ is vertical and $P$ lies on $P_1P_2$ iff $x=x_1$.
        \item If $x_1\neq x_2$, then the slope of $P_1P_2$ $m_{P_1P_2}=\frac{y_2-y_1}{x_2-x_1}$. Thus, $P$ lies on $P_1P_2$ iff $P=P_1$ or, for the line $PP_1$ through $P$ and $P_1$, $m_{P_1P_2}=m_{PP_1}=\frac{y-y_1}{x-x_1}$. In other words, the coordinates $x,y$ of $P$ must satisfy $y-y_1=m_{P_1P_2}(x-x_1)$.
    \end{itemize}
    \item \cite{bib:Thomas} calls the above equation the \textbf{point-slope form}.
    \item \textbf{Variable}: \dq{A symbol, such as $x$, which may take on any value in some set of numbers}{10}
    \item \textbf{Slope-intercept form}: $y=mx+b$.
    \item \textbf{General form}: $Ax+By+C=0$.
    \begin{itemize}
        \item Such an equation (one that contains only first powers of $x$ and $y$ and constants) is said to be \textbf{linear in \emph{x} and \emph{y}}.
        \item \dq{Every straight line in the plane is represented by a linear equation and, conversely, every linear equation represents a straight line}{10}
    \end{itemize}
    \item \textbf{\emph{y}-intercept}: The constant $b$ in the above equation.
    \item Let $L$ be a line with the equation $Ax+By+C=0$. The shortest distance $d$ from a point $P_1(x_1,y_1)$ not on $L$ to $L$ is given by
    \begin{equation*}
        d = \frac{|Ax_1+By_1+C|}{\sqrt{A^2+B^2}}
    \end{equation*}
    \begin{itemize}
        \item Derive by finding a line perpendicular to $L$ through $P_1$.
    \end{itemize}
\end{itemize}



\section{Functions and Graphs}
\begin{itemize}
    \item \textbf{Domain} (of a variable $x$): \dq{The set of numbers over which $x$ may vary}{12}
    \item Defines \textbf{open intervals}, \textbf{half-open intervals}, and \textbf{closed intervals}.
    \begin{figure}[h!]
        \centering
        \begin{tikzpicture}[scale=1.7]
            \footnotesize
            \fill [grx] (0,0) rectangle (2,1.2);
            \fill [yly] (0.7,0.2)
                to[out=0,in=-90] (1.4,0.7)
                to[out=90,in=0] (1.1,1)
                to[out=180,in=90] (0.4,0.5)
                to[out=-90,in=180] cycle
            ;
            \node (x) [circle,fill,inner sep=1.5pt,label={left:$x$}] at (1.1,0.55) {};
            \node at (0.15,1.05) {$X$};
            \node at (1.2,0.8) {$D_f$};

            \begin{scope}[xshift=2.8cm]
                \fill [grx] (0,0) rectangle (2,1.2);
                \fill [yly] (0.9,0.2)
                    to[out=0,in=-90] (1.6,0.7)
                    to[out=90,in=0] (1,1.05)
                    to[out=180,in=90] (0.4,0.5)
                    to[out=-90,in=180] cycle
                ;
                \node (y) [circle,fill,inner sep=1.5pt,label={right:$y$}] at (0.8,0.5) {}
                    edge [latex-,thick,bend left=20] node[above]{$f$} (x)
                ;
                \node at (0.15,1.05) {$Y$};
                \node at (1.2,0.8) {$R_f$};
            \end{scope}
        \end{tikzpicture}
        \caption{A function $f$ maps the domain $D_f$ onto the range $R_f$. The image of $x$ is $y=f(x)$.}
        \label{fig:mapping}
    \end{figure}
    \item \textbf{Function}: For two nonempty sets $X,Y$, the collection $f$ of ordered pairs $(x,y)$ with $x\in X$ and $y\in Y$ that assigns to every $x\in X$ a unique $y\in Y$. \emph{Also known as} \textbf{mapping} (from $X$ to $Y$), $\bm{y=f(x)}$, $\bm{f:x\to y}$\footnote{"eff sends ex into wy"}.
    \begin{itemize}
        \item When using the latter notation, it is understood that the domain is $\R$ unless this is impossible (e.g., $f:x\to\frac{1}{x}$ must exclude 0 from the domain).
    \end{itemize}
    \item \textbf{Domain} (of a function $f$): \dq{The collection of all first elements $x$ of the pairs $(x,y)$ in $f$}{13} \emph{Also known as} $\bm{D_f}$.
    \item \textbf{Range} (of a function $f$): \dq{The set of all second elements $y$ of the pairs $(x,y)$ in $f$}{13} \emph{Also known as} $\bm{R_f}$.
    \item \textbf{Image} (of $x$): The value $y$ to which a function maps $x$.
    \item \cite{bib:Thomas} considers functions from the reals to the reals, but also more abstract functions.
    \begin{itemize}
        \item For example, it considers the function from all triangles (a set of decidedly nonnumerical objects) to their enclosed areas (the set of positive real numbers).
    \end{itemize}
    \begin{figure}[h!]
        \centering
        \begin{tikzpicture}
            \footnotesize
            \draw [-stealth] (-2.8,0) -- (2.8,0) node[right]{$x$};
            \draw [-stealth] (0,-0.5) -- (0,4.8) node[above]{$y$};
            \foreach \x in {-2,-1,1,2} {
                \draw (\x,0.1) -- ++(0,-0.2) node[below]{$\x$};
            }
            \foreach \y in {1,2,3,4} {
                \draw (0.1,\y) -- ++(-0.2,0) node[left]{$\y$};
            }
            \node [below left=1mm] {$0$};

            \draw [ylx,thick] plot [domain=-2:2,smooth] (\x,{\x*\x}) node[above,black]{$y=x^2$, $-2\leq x\leq 2$};
        \end{tikzpicture}
        \caption{Graph of a function.}
        \label{fig:graphOfFunction}
    \end{figure}
    \item \textbf{Graph} (of a function): \dq{The set of points which correspond to members of the function}{14}
    \begin{itemize}
        \item For example, let $X$ be the closed interval $[-2,2]$. To each $x\in X$, assign the number $x^2$. This describes the function
        \begin{equation*}
            f = \left\{ (x,y):-2\leq x\leq 2,\ y=x^2 \right\}
        \end{equation*}
        The graph of $f$ can be seen in Figure \ref{fig:graphOfFunction}.
    \end{itemize}
    \item \textbf{Independent variable}: The first variable $x$ in the ordered pair $(x,y)$. \emph{Also known as} \textbf{argument}.
    \item \textbf{Dependent variable}: The second variable $y$ in the ordered pair $(x,y)$.
    \item \textbf{Real-valued function of a real variable}: \dq{A function $f$ whose domain and range are sets of real numbers}{14}
    \begin{itemize}
        \item As a general rule, \emph{function} indicates a real-valued function of a real variable for the first seven chapters of \cite{bib:Thomas}.
    \end{itemize}
    \item $f$ can be represented by\dots
    \begin{itemize}
        \item A table of corresponding values (this will be incomplete, though).
        \item Corresponding numerical scales, as on a slide rule (this will be incomplete, though).
        \item A simple formula, such as $f(x)=x^2$ (this may be less exact than ordered pairs, but it is more easily understood/applicable/complete).
        \item A graph (for any value $x$ in the domain, begin $x$ units from the origin along the $x$-axis, move vertically until intersecting the curve, and then move horizontally until intersecting the image $y$ on the $y$-axis).
    \end{itemize}
    \item Some mappings cannot be expressed in terms of algebraic operations on $x$.
    \begin{itemize}
        \item For example, the \textbf{greatest-integer function} \dq{maps any real number $x$ onto that unique integer which is the largest among all integers that are less than or equal to $x$}{15}
        \begin{itemize}
            \item The image of $x$ is represented by $[x]$, and the function by $f:x\to[x]$.
            \item An example of a \textbf{step function}.
            \item It exhibits points of \textbf{discontinuity}.
        \end{itemize}
    \end{itemize}
    \item Note: The fact that a one-to-one mapping exists between the points in the interval $(0,1]$ and $[1,\infty)$ (namely, $f:x\to\frac{1}{x}$) proves that there are equally many points in both intervals.
    \item The absolute value function can be geometrically interpreted in the context of distance from a point. As such, it is useful in describing \textbf{neighborhoods}.
    \begin{figure}[h!]
        \centering
        \begin{tikzpicture}[
            scale=1.3,
            every node/.style={black}
        ]
            \footnotesize
            \draw [ylx,line width=2.5pt] (-1,0) node[above left]{$c-h$} -- node[above left]{$c$} (1,0) node[above right]{$c+h$};
            \draw (-2,0) -- (2,0);
            \draw circle (1cm);
            \draw (0,0) -- node[above]{$h$} (35:1);
        \end{tikzpicture}
        \caption{The symmetric neighborhood $N_h(c)$, centered at $c$, with radius $h$.}
        \label{fig:neighborhood+radius}
    \end{figure}
    \item \textbf{Symmetric neighborhood} (of a point $c$): \dq{The open interval $(c-h,c+h)$, where $h$ may be any positive number}{17} \emph{Also known as} $\bm{N_h(c)}$.
    \item \textbf{Radius} (of a symmetric neighborhood): The value $h$ (see Figure \ref{fig:neighborhood+radius}).
    \item \textbf{Neighborhood} (of a point $c$): The open interval $(c-h,c+k)$, where $h,k$ may be any positive numbers.
    \begin{itemize}
        \item Like requiring that $|x-c|$ is small.
    \end{itemize}
    \item \textbf{Deleted neighborhood} (of a point $c$): \dq{A neighborhood of $c$ from which $c$ itself has been removed}{17}
    \begin{itemize}
        \item Like requiring that $|x-c|>0$.
    \end{itemize}
    \item \textbf{Intersection} (of the neighborhoods $(c-h_1,c+k_1)$ and $(c-h_2,c+k_2)$): The neighborhood $(c-h,c+k)$, where $h=\min(h_1,h_2)$ and $k=\min(k_1,k_2)$.
    \begin{itemize}
        \item \dq{The intersection of two neighborhoods of $c$ is a neighborhood of $c$, and the intersection of two deleted neighborhoods of $c$ is a deleted neighborhood of $c$}{18}
    \end{itemize}
    \item Let $A$ be a neighborhood of $c$. Then denote the deleted neighborhood equivalent to $A$ with $c$ removed by $A^-$.
    \item \marginnote{7/6:}\dq{A function is determined by the domain and by any rule that tells what image in the range is to be associated with each element of the domain}{18}
    \begin{itemize}
        \item Thus, we can think of a "function machine" that takes in elements of the domain and computes the image based on the rule.
        \item \cite{bib:Thomas} visualizes function machines as flow charts.
        \item In theory, a function machine could store every pair $(x,y)$ in its memory to be recalled later. Since machines have limited memory, a fractional set of pairs could also be stored and the values in between calculated by interpolation.
        \item In practice, though, calculating as we go is usually best.
    \end{itemize}
    \item Two restrictions should be inferred to apply to the domain of a function, even if they are unstated: First, never divide by 0. Second, do not consider complex outputs (yet).
    \item Sometimes we have functions of more than one independent variable.
    \begin{itemize}
        \item For example, the volume $v=\frac{1}{3}\pi r^2h$ of a right circular cone is uniquely determined only when $r,h$ are given definite, positive, nonzero values.
        \item \dq{Its domain is the set of all pairs $(r,h)$ with $r>0$, $h>0$. Its range is the set of positve numbers $v>0$}{20}
        \item $r,h$ are independent variables. $v$ is a dependent variable.
    \end{itemize}
    \item \dq{More generally, suppose that some quantity $y$ is uniquely determined by $n$ other quantities $x_1,x_2,$ $x_3,\dots,x_n$. The set of all ordered $(n+1)$-tuples $(x_1,x_2,x_3,\dots,x_n,y)$ that can be obtained by substituting permissible values of the variables $x_1,x_2,\dots,x_n$ and the corresponding values of $y$ is a function whose domain is the set of all allowable $n$-tuples $(x_1,x_2,x_3,\dots,x_n)$ and whose range is the set of all possible values of $y$ corresponding to this domain. If values can be assigned independently to each of the $x$'s, we call them independent variables and say that $y$ is a function of the $x$'s. We also write $y=f(x_1,x_2,x_3,\dots,x_n)$ to indicate that $y$ is a function of the $n$ $x$'s, just as we write $y=f(x)$ to indicate that $y$ is a function of one independent variable $x$}{20}
    \begin{itemize}
        \item Note, though, that functions of a single variable will be the primary concern of this book.
    \end{itemize}
    \item \textbf{Signum function}: The function
    \begin{equation*}
        \text{sgn}\, x =
        \begin{cases}
            -1 & x<0\\
            0 & x=0\\
            1 & x>0
        \end{cases}
    \end{equation*}
\end{itemize}




\end{document}