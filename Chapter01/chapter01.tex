\documentclass[../main.tex]{subfiles}

\pagestyle{main}
\renewcommand{\chaptermark}[1]{\markboth{\chaptername\ \thechapter: #1}{}}

\begin{document}




\chapter{The Rate of Change of a Function}
\section{Introduction}
\begin{itemize}
    \item \marginnote{7/3:}Discusses the importance of calculus, when it should be used, and why one should study it.
    \item \textbf{Analytic geometry}: \dq{Uses algebraic methods and equations to study geometric problems. Conversely, it permits us to visualize algebraic equations in terms of geometric curves}{2}
\end{itemize}



\section{Coordinates}
\begin{itemize}
    \item \dq{The basic idea in analytic geometry is the establishment of a one-to-one correspondence between the points of a plane on the one hand and pairs of numbers $(x,y)$ on the other hand}{2}
    \item Such a correspondence is most commonly established as follows.
    \begin{figure}[h!]
        \centering
        \begin{tikzpicture}
            \footnotesize
            \draw [-stealth] (-2.8,0) -- (2.8,0) node[right]{$x$};
            \draw [-stealth] (0,-3) -- (0,3) node[above]{$y$};

            \draw (-2.1,0.1) -- ++(0,-0.2) node[below]{$-a$};
            \draw (1,0.1) -- ++(0,-0.2) node[below]{$+1$};
            \draw (2.1,0.1) -- ++(0,-0.2) node[below]{$+a$};
            \draw (0.1,-1.7) -- ++(-0.2,0) node[left]{$-b$};
            \draw (0.1,0.8) -- ++(-0.2,0) node[left]{$+1$};
            \draw (0.1,1.7) -- ++(-0.2,0) node[left]{$+b$};

            \draw [ylx,thick] (2.1,0) -- ++(0,1.7) node[right,black]{$P(a,b)$} -- ++(-2.1,0);

            \node [below left] {$O$};
            \node at (1.8,2.4) {$I(+,+)$};
            \node at (-1.8,2.4) {$II(-,+)$};
            \node at (-1.8,-2.4) {$III(-,-)$};
            \node at (1.8,-2.4) {$IV(+,-)$};
        \end{tikzpicture}
        \caption{Cartesian coordinates.}
        \label{fig:cartesianPlane}
    \end{figure}
    \begin{itemize}
        \item \dq{A horizontal line in the plane, extending indefinitely to the left and to the right, is chosen as the $x$-axis or axis of \textbf{abscissas}. A reference point $O$ on this line and a unit of length are then chosen. The axis is scaled off in terms of this unit of length in such a way that the number zero is attached to $O$, the number $+a$ is attached to the point which is $a$ units to the right of $O$, and $-a$ is attached to the symmetrically located point to the left of $O$. In this way, a one-to-one correspondence is established between points of the $x$-axis and the set of all \textbf{real numbers}}{2-3}
        \item \dq{Now through $O$ take a second, vertical line in the plane, extending indefinitely up and down. This line becomes the $y$-axis, or axis of \textbf{ordinates}. The unit of length used to represent $+1$ on the $y$-axis need not be the same as the unit of length used to represent $+1$ on the $x$-axis. The $y$-axis is scaled off in terms of the unit of length adopted for it, with the positive number $+b$ attached to the point $b$ units above $O$ and negative number $-b$ attached to the symmetrically located point $b$ units below $O$}{3}
        \item \dq{If a line parallel to the $y$-axis is drawn through the point marked $a$ on the $x$-axis, and a line parallel to the $x$-axis is drawn through the point marked $b$ on the $y$-axis, their point of intersection $P$ is to be labeled $P(a,b)$. Thus, given the pair of real numbers $a$ and $b$, we find one and only one point with abscissa $a$ and ordinate $b$, and this point we denote by $P(a,b)$}{3}
        \item \dq{Conversely, if we start with any point $P$ in the plane, we may draw lines through it parallel to the coordinate axes. If these lines intersect the $x$-axis at $a$ and the $y$-axis at $b$, we then regard the pair of numbers $(a,b)$ as corresponding to the point $P$. We say that the coordinates of $P$ are $(a,b)$}{3}
        \item \dq{The two axes divide the plane into four quadrants, called the first quadrant, second quadrant, and so on, and labeled I, II, III, IV in [Figure \ref{fig:cartesianPlane}]. Points in the first quadrant have both coordinates positive, and in the second quadrant the $x$-coordinate (abscissa) is negative and the $y$-coordinate (ordinate) is positive. The notations $(-,-)$ and $(+,-)$ in quadrants III and IV of [Figure \ref{fig:cartesianPlane}] represent the signs of the coordinates of points in these quadrants}{3}
    \end{itemize}
\end{itemize}



\section{Increments}
\begin{itemize}
    \item \textbf{Increments}: The values $\Delta x = x_2-x_1$ and $\Delta y = y_2-y_1$ concerning a particle, the initial position of which is $P_1(x_1,y_1)$ and the terminal position of which is $P_2(x_2,y_2)$.
    \item If the unit of measurement for both axes is the same, then we may express distances in the plane in terms of this unit using the Pythagorean theorem.
\end{itemize}



\section{Slope of a Straight Line}
\begin{itemize}
    \item Let $L$ be a straight line not parallel to the $y$-axis intersecting distinct points $P_1(x_1,y_1)$ and $P_2(x_2,y_2)$. Then $L$ has a \textbf{rise}, \textbf{run}, and \textbf{slope}.
    \item \textbf{Rise}: The increment $\Delta y$.
    \item \textbf{Run}: The increment $\Delta x$.
    \item \textbf{Slope}: The rate of rise per run $m=\frac{\text{rise}}{\text{run}}=\frac{\Delta y}{\Delta x}=\frac{y_2-y_1}{x_2-x_1}$. \emph{Also known as} \textbf{inclination}.
    \begin{figure}[h!]
        \centering
        \begin{subfigure}[b]{0.3\linewidth}
            \centering
            \begin{tikzpicture}
                \footnotesize
                \coordinate (A) at (0,0);
                
                \draw [-stealth,name path=x axis] (-2,0) -- (1.3,0) node[right]{$x$};
                \draw [-stealth,name path=y axis] (0,-0.4) -- (0,2) node[above]{$y$};

                \draw [ylx,thick,name path=line] (-1.9,-0.2) -- (1.2,1.9);

                \path[name intersections={of=x axis and line,by=B}];
                \path[name intersections={of=y axis and line,by=C}];
                \pic[draw,->,pic text={$\phi$},angle eccentricity=1.3]{angle};

                \node [below left] {$O$};
            \end{tikzpicture}
            \caption{Negative $x$-intercept.}
            \label{fig:angleOfInclinationa}
        \end{subfigure}
        \begin{subfigure}[b]{0.3\linewidth}
            \centering
            \begin{tikzpicture}
                \footnotesize
                \coordinate (A) at (2,0);
                
                \draw [-stealth,name path=x axis] (-0.5,0) -- (2.5,0) node[right]{$x$};
                \draw [-stealth,name path=y axis] (0,-0.4) -- (0,2) node[above]{$y$};

                \draw [ylx,thick,name path=line] (1.9,-0.2) -- (-0.4,1.9);

                \path[name intersections={of=x axis and line,by=B}];
                \path[name intersections={of=y axis and line,by=C}];
                \pic[draw,->,pic text={$\phi$},angle eccentricity=1.3]{angle};

                \node [below left] {$O$};
            \end{tikzpicture}
            \caption{Positive $x$-intercept.}
            \label{fig:angleOfInclinationb}
        \end{subfigure}
        \caption{The slope and the angle of inclination.}
        \label{fig:angleOfInclination}
    \end{figure}
    \begin{itemize}
        \item If we chose different distinct points, the slope would be same because the triangles in the Cartesian plane would be similar.
        \item $\Delta y$ is proportional to $\Delta x$ with $m$ as the proportionality factor.
        \item On interpolation: If we're given the values of a function at $(x_1,y_1)$ and $(x_2,y_2)$, then we may approximate the function by a straight line $L$ passing through those two points and approximate the value $f(x)$ for any $x_1\leq x\leq x_2$.
        \item If the scales on both axes are equal, then the slope of $L$ is equal to the tangent of the \textbf{angle of inclination} that $L$ makes with the positive $x$-axis. That is, $m=\tan\phi$ (see Figure \ref{fig:angleOfInclination}).
    \end{itemize}
    \item \textbf{Parallel} (lines): Two lines with equal inclinations ($m_1=m_2$).
    \item \textbf{Perpendicular} (lines): Two lines with inclinations that differ by $90^\circ$ ($m_1=-\frac{1}{m_2}$).
    \begin{itemize}
        \item Note that we can prove the relation between the slopes using the angles of inclination as follows.
        \begin{align*}
            m_1 &= \tan\phi_1\\
            &= \tan\left( \phi_2+90^\circ \right)\\
            &= -\cot\phi_2\\
            &= -\frac{1}{\tan\phi_2}\\
            &= -\frac{1}{m_2}
        \end{align*}
    \end{itemize}
\end{itemize}



\section{Equations of a Straight Line}
\begin{itemize}
    \item \marginnote{7/5:}How do you know if $P(x,y)$ is a point on the line $P_1P_2$ through distinct points $P_1(x_1,y_1)$ and $P_2(x_2,y_2)$?
    \begin{itemize}
        \item If $x_1=x_2$, then $P_1P_2$ is vertical and $P$ lies on $P_1P_2$ iff $x=x_1$.
        \item If $x_1\neq x_2$, then the slope of $P_1P_2$ $m_{P_1P_2}=\frac{y_2-y_1}{x_2-x_1}$. Thus, $P$ lies on $P_1P_2$ iff $P=P_1$ or, for the line $PP_1$ through $P$ and $P_1$, $m_{P_1P_2}=m_{PP_1}=\frac{y-y_1}{x-x_1}$. In other words, the coordinates $x,y$ of $P$ must satisfy $y-y_1=m_{P_1P_2}(x-x_1)$.
    \end{itemize}
    \item \cite{bib:Thomas} calls the above equation the \textbf{point-slope form}.
    \item \textbf{Variable}: \dq{A symbol, such as $x$, which may take on any value in some set of numbers}{10}
    \item \textbf{Slope-intercept form}: $y=mx+b$.
    \item \textbf{General form}: $Ax+By+C=0$.
    \begin{itemize}
        \item Such an equation (one that contains only first powers of $x$ and $y$ and constants) is said to be \textbf{linear in \emph{x} and \emph{y}}.
        \item \dq{Every straight line in the plane is represented by a linear equation and, conversely, every linear equation represents a straight line}{10}
    \end{itemize}
    \item \textbf{\emph{y}-intercept}: The constant $b$ in the above equation.
    \item Let $L$ be a line with the equation $Ax+By+C=0$. The shortest distance $d$ from a point $P_1(x_1,y_1)$ not on $L$ to $L$ is given by
    \begin{equation*}
        d = \frac{|Ax_1+By_1+C|}{\sqrt{A^2+B^2}}
    \end{equation*}
    \begin{itemize}
        \item Derive by finding a line perpendicular to $L$ through $P_1$.
    \end{itemize}
\end{itemize}



\section{Functions and Graphs}
\begin{itemize}
    \item \textbf{Domain} (of a variable $x$): \dq{The set of numbers over which $x$ may vary}{12}
    \item Defines \textbf{open intervals}, \textbf{half-open intervals}, and \textbf{closed intervals}.
    \begin{figure}[h!]
        \centering
        \begin{tikzpicture}[scale=1.7]
            \footnotesize
            \fill [grx] (0,0) rectangle (2,1.2);
            \fill [yly] (0.7,0.2)
                to[out=0,in=-90] (1.4,0.7)
                to[out=90,in=0] (1.1,1)
                to[out=180,in=90] (0.4,0.5)
                to[out=-90,in=180] cycle
            ;
            \node (x) [circle,fill,inner sep=1.5pt,label={left:$x$}] at (1.1,0.55) {};
            \node at (0.15,1.05) {$X$};
            \node at (1.2,0.8) {$D_f$};

            \begin{scope}[xshift=2.8cm]
                \fill [grx] (0,0) rectangle (2,1.2);
                \fill [yly] (0.9,0.2)
                    to[out=0,in=-90] (1.6,0.7)
                    to[out=90,in=0] (1,1.05)
                    to[out=180,in=90] (0.4,0.5)
                    to[out=-90,in=180] cycle
                ;
                \node (y) [circle,fill,inner sep=1.5pt,label={right:$y$}] at (0.8,0.5) {}
                    edge [latex-,thick,bend left=20] node[above]{$f$} (x)
                ;
                \node at (0.15,1.05) {$Y$};
                \node at (1.2,0.8) {$R_f$};
            \end{scope}
        \end{tikzpicture}
        \caption{A function $f$ maps the domain $D_f$ onto the range $R_f$. The image of $x$ is $y=f(x)$.}
        \label{fig:mapping}
    \end{figure}
    \item \textbf{Function}: For two nonempty sets $X,Y$, the collection $f$ of ordered pairs $(x,y)$ with $x\in X$ and $y\in Y$ that assigns to every $x\in X$ a unique $y\in Y$. \emph{Also known as} \textbf{mapping} (from $X$ to $Y$), $\bm{y=f(x)}$, $\bm{f:x\to y}$\footnote{"eff sends ex into wy"}.
    \begin{itemize}
        \item When using the latter notation, it is understood that the domain is $\R$ unless this is impossible (e.g., $f:x\to\frac{1}{x}$ must exclude 0 from the domain).
    \end{itemize}
    \item \textbf{Domain} (of a function $f$): \dq{The collection of all first elements $x$ of the pairs $(x,y)$ in $f$}{13} \emph{Also known as} $\bm{D_f}$.
    \item \textbf{Range} (of a function $f$): \dq{The set of all second elements $y$ of the pairs $(x,y)$ in $f$}{13} \emph{Also known as} $\bm{R_f}$.
    \item \textbf{Image} (of $x$): The value $y$ to which a function maps $x$.
    \item \cite{bib:Thomas} considers functions from the reals to the reals, but also more abstract functions.
    \begin{itemize}
        \item For example, it considers the function from all triangles (a set of decidedly nonnumerical objects) to their enclosed areas (the set of positive real numbers).
    \end{itemize}
    \begin{figure}[h!]
        \centering
        \begin{tikzpicture}
            \footnotesize
            \draw [-stealth] (-2.8,0) -- (2.8,0) node[right]{$x$};
            \draw [-stealth] (0,-0.5) -- (0,4.8) node[above]{$y$};
            \foreach \x in {-2,-1,1,2} {
                \draw (\x,0.1) -- ++(0,-0.2) node[below]{$\x$};
            }
            \foreach \y in {1,2,3,4} {
                \draw (0.1,\y) -- ++(-0.2,0) node[left]{$\y$};
            }
            \node [below left=1mm] {$0$};

            \draw [ylx,thick] plot [domain=-2:2,smooth] (\x,{\x*\x}) node[above,black]{$y=x^2$, $-2\leq x\leq 2$};
        \end{tikzpicture}
        \caption{Graph of a function.}
        \label{fig:graphOfFunction}
    \end{figure}
    \item \textbf{Graph} (of a function): \dq{The set of points which correspond to members of the function}{14}
    \begin{itemize}
        \item For example, let $X$ be the closed interval $[-2,2]$. To each $x\in X$, assign the number $x^2$. This describes the function
        \begin{equation*}
            f = \left\{ (x,y):-2\leq x\leq 2,\ y=x^2 \right\}
        \end{equation*}
        The graph of $f$ can be seen in Figure \ref{fig:graphOfFunction}.
    \end{itemize}
    \item \textbf{Independent variable}: The first variable $x$ in the ordered pair $(x,y)$. \emph{Also known as} \textbf{argument}.
    \item \textbf{Dependent variable}: The second variable $y$ in the ordered pair $(x,y)$.
    \item \textbf{Real-valued function of a real variable}: \dq{A function $f$ whose domain and range are sets of real numbers}{14}
    \begin{itemize}
        \item As a general rule, \emph{function} indicates a real-valued function of a real variable for the first seven chapters of \cite{bib:Thomas}.
    \end{itemize}
    \item $f$ can be represented by\dots
    \begin{itemize}
        \item A table of corresponding values (this will be incomplete, though).
        \item Corresponding numerical scales, as on a slide rule (this will be incomplete, though).
        \item A simple formula, such as $f(x)=x^2$ (this may be less exact than ordered pairs, but it is more easily understood/applicable/complete).
        \item A graph (for any value $x$ in the domain, begin $x$ units from the origin along the $x$-axis, move vertically until intersecting the curve, and then move horizontally until intersecting the image $y$ on the $y$-axis).
    \end{itemize}
    \item Some mappings cannot be expressed in terms of algebraic operations on $x$.
    \begin{itemize}
        \item For example, the \textbf{greatest-integer function} \dq{maps any real number $x$ onto that unique integer which is the largest among all integers that are less than or equal to $x$}{15}
        \begin{itemize}
            \item The image of $x$ is represented by $[x]$, and the function by $f:x\to[x]$.
            \item An example of a \textbf{step function}.
            \item It exhibits points of \textbf{discontinuity}.
        \end{itemize}
    \end{itemize}
    \item Note: The fact that a one-to-one mapping exists between the points in the interval $(0,1]$ and $[1,\infty)$ (namely, $f:x\to\frac{1}{x}$) proves that there are equally many points in both intervals.
    \item The absolute value function can be geometrically interpreted in the context of distance from a point. As such, it is useful in describing \textbf{neighborhoods}.
    \begin{figure}[h!]
        \centering
        \begin{tikzpicture}[
            scale=1.3,
            every node/.style={black}
        ]
            \footnotesize
            \draw [ylx,line width=2.5pt] (-1,0) node[above left]{$c-h$} -- node[above left]{$c$} (1,0) node[above right]{$c+h$};
            \draw (-2,0) -- (2,0);
            \draw circle (1cm);
            \draw (0,0) -- node[above]{$h$} (35:1);
        \end{tikzpicture}
        \caption{The symmetric neighborhood $N_h(c)$, centered at $c$, with radius $h$.}
        \label{fig:neighborhood+radius}
    \end{figure}
    \item \textbf{Symmetric neighborhood} (of a point $c$): \dq{The open interval $(c-h,c+h)$, where $h$ may be any positive number}{17} \emph{Also known as} $\bm{N_h(c)}$.
    \item \textbf{Radius} (of a symmetric neighborhood): The value $h$ (see Figure \ref{fig:neighborhood+radius}).
    \item \textbf{Neighborhood} (of a point $c$): The open interval $(c-h,c+k)$, where $h,k$ may be any positive numbers.
    \begin{itemize}
        \item Like requiring that $|x-c|$ is small.
    \end{itemize}
    \item \textbf{Deleted neighborhood} (of a point $c$): \dq{A neighborhood of $c$ from which $c$ itself has been removed}{17}
    \begin{itemize}
        \item Like requiring that $|x-c|>0$.
    \end{itemize}
    \item \textbf{Intersection} (of the neighborhoods $(c-h_1,c+k_1)$ and $(c-h_2,c+k_2)$): The neighborhood $(c-h,c+k)$, where $h=\min(h_1,h_2)$ and $k=\min(k_1,k_2)$.
    \begin{itemize}
        \item \dq{The intersection of two neighborhoods of $c$ is a neighborhood of $c$, and the intersection of two deleted neighborhoods of $c$ is a deleted neighborhood of $c$}{18}
    \end{itemize}
    \item Let $A$ be a neighborhood of $c$. Then denote the deleted neighborhood equivalent to $A$ with $c$ removed by $A^-$.
    \item \marginnote{7/6:}\dq{A function is determined by the domain and by any rule that tells what image in the range is to be associated with each element of the domain}{18}
    \begin{itemize}
        \item Thus, we can think of a "function machine" that takes in elements of the domain and computes the image based on the rule.
        \item \cite{bib:Thomas} visualizes function machines as flow charts.
        \item In theory, a function machine could store every pair $(x,y)$ in its memory to be recalled later. Since machines have limited memory, a fractional set of pairs could also be stored and the values in between calculated by interpolation.
        \item In practice, though, calculating as we go is usually best.
    \end{itemize}
    \item Two restrictions should be inferred to apply to the domain of a function, even if they are unstated: First, never divide by 0. Second, do not consider complex outputs (yet).
    \item Sometimes we have functions of more than one independent variable.
    \begin{itemize}
        \item For example, the volume $v=\frac{1}{3}\pi r^2h$ of a right circular cone is uniquely determined only when $r,h$ are given definite, positive, nonzero values.
        \item \dq{Its domain is the set of all pairs $(r,h)$ with $r>0$, $h>0$. Its range is the set of positve numbers $v>0$}{20}
        \item $r,h$ are independent variables. $v$ is a dependent variable.
    \end{itemize}
    \item \dq{More generally, suppose that some quantity $y$ is uniquely determined by $n$ other quantities $x_1,x_2,$ $x_3,\dots,x_n$. The set of all ordered $(n+1)$-tuples $(x_1,x_2,x_3,\dots,x_n,y)$ that can be obtained by substituting permissible values of the variables $x_1,x_2,\dots,x_n$ and the corresponding values of $y$ is a function whose domain is the set of all allowable $n$-tuples $(x_1,x_2,x_3,\dots,x_n)$ and whose range is the set of all possible values of $y$ corresponding to this domain. If values can be assigned independently to each of the $x$'s, we call them independent variables and say that $y$ is a function of the $x$'s. We also write $y=f(x_1,x_2,x_3,\dots,x_n)$ to indicate that $y$ is a function of the $n$ $x$'s, just as we write $y=f(x)$ to indicate that $y$ is a function of one independent variable $x$}{20}
    \begin{itemize}
        \item Note, though, that functions of a single variable will be the primary concern of this book.
    \end{itemize}
    \item \textbf{Signum function}: The function
    \begin{equation*}
        \text{sgn}\, x =
        \begin{cases}
            -1 & x<0\\
            0 & x=0\\
            1 & x>0
        \end{cases}
    \end{equation*}
\end{itemize}



\section{Ways of Combining Functions}
\begin{itemize}
    \item \marginnote{7/7:}The domains of the sum, product, difference, and quotient of two functions $f$ and $g$ are the intersections of $D_f$ and $D_g$.
    \begin{itemize}
        \item Note that for the quotient, we must also exclude points where $g(x)=0$.
    \end{itemize}
    \item There is a distinction between $x\cdot\frac{1}{x}$ and $1$ (namely, the fact that the latter includes 0 in its domain while the former excludes it).
    \item Translation.
    \begin{figure}[h!]
        \centering
        \begin{tikzpicture}[
            scale=1.2,
            every node/.style={black,text height=1.5ex,text depth=0.25ex}
        ]
            \footnotesize
            \draw [-stealth] (-0.4,0) -- (5.2,0) node[right]{$x$};
            \draw [-stealth] (0,-0.4) -- (0,3) node[above]{$y$};

            \draw [ylx,thick,densely dashed] (0.4,1.5) node[above right,yshift=3.5mm]{$f$} parabola [bend at end] (2,2.7);
            \draw [ylx,semithick,densely dashed] (0.4,1.5) -- (0.4,0) node[below]{$a$};
            \draw [ylx,semithick,densely dashed] (1,2.23) -- node[right=-0.5mm]{$f(x-c)$} (1,0) node[below]{$x-c$};
            \draw [ylx,semithick,densely dashed] (2,2.7) -- (2,0) node[below]{$b$};

            \begin{scope}[xshift=2.3cm]
                \draw [ylx,thick] (0.4,1.5) node[above right,yshift=3.5mm]{$g$} parabola [bend at end] (2,2.7);
                \draw [ylx,semithick] (0.4,1.5) -- (0.4,0) node[below]{$a+c$};
                \draw [ylx,semithick] (1,2.23) -- node[right=-0.5mm]{$g(x)$} (1,0) node[below]{$x$};
                \draw [ylx,semithick] (2,2.7) -- (2,0) node[below]{$b+c$};
            \end{scope}

            \draw [-latex] (1.4,-0.4) to[bend right=20] (3,-0.4);

            \node [below left] {$O$};
        \end{tikzpicture}
        \caption{Translation.}
        \label{fig:translation}
    \end{figure}
    \begin{itemize}
        \item \dq{Take the ordinate of $f$ at $x-c$ and shift it to the right $c$ units to get the ordinate of $g$ at $x$}{23}
        \item $g(x)=f(x-c)$ implies that the graph of $g$ is that of $f$ translated $c$ units to the right.
        \item If $D_f=[a,b]$ ($f(x)$ is only defined when $a\leq x\leq b$), then $f(x-c)$ is only defined when $a\leq x-c\leq b$. This implies that $g(x)$ is only defined when $a+c\leq x\leq b+c$; hence, $D_g=[a+c,b+c]$.
    \end{itemize}
    \item Change of scale.
    \begin{figure}[h!]
        \centering
        \begin{subfigure}[b]{0.49\linewidth}
            \centering
            \begin{tikzpicture}[scale=0.9]
                \footnotesize
                \draw [-stealth] (-4,0) -- (4,0) node[right]{$x$};
                \draw [-stealth] (0,-2.6) -- (0,2.6) node[above]{$y$};
                
                \draw [yly,thick] plot [domain=-pi:pi,smooth] (\x,{sin(\x r)});
                \draw [ylx,thick] plot [domain=-1:1,smooth] (\x,{sin(pi*\x r)});

                \foreach \x/\text in {-pi/-\pi,-0.5*pi/-\frac{\pi}{2},-1,1,0.5*pi/\frac{\pi}{2},pi/\pi} {
                    \draw (\x,0.1) -- ++(0,-0.2) node[below]{$\text$};
                }
                \foreach \y in {-2,-1,1,2} {
                    \draw (0.1,\y) -- ++(-0.2,0) node[left]{$\y$};
                }

                \node [below right] {$0$};
                \node [above right] at (0.5,1) {$y=g(x)=\sin\pi x$};
                \node [below left] at (-0.5*pi,-1) {$y=f(x)=\sin x$};
            \end{tikzpicture}
            \caption{Horizontal scaling.}
            \label{fig:scalinga}
        \end{subfigure}
        \begin{subfigure}[b]{0.49\linewidth}
            \centering
            \begin{tikzpicture}[scale=0.9]
                \footnotesize
                \draw [-stealth] (-4,0) -- (4,0) node[right]{$x$};
                \draw [-stealth] (0,-2.6) -- (0,2.6) node[above]{$y$};
                
                \draw [yly,thick] plot [domain=-pi:pi,smooth] (\x,{sin(\x r)});
                \draw [ylx,thick] plot [domain=-pi:pi,smooth] (\x,{2*sin(\x r)});

                \foreach \x/\text in {-pi/-\pi,-0.5*pi/-\frac{\pi}{2},0.5*pi/\frac{\pi}{2},pi/\pi} {
                    \draw (\x,0.1) -- ++(0,-0.2) node[below]{$\text$};
                }
                \foreach \y in {-2,-1,1,2} {
                    \draw (0.1,\y) -- ++(-0.2,0) node[left]{$\y$};
                }

                \node [below right] {$0$};
                \node [above,fill=white,inner sep=1pt] at (0.5*pi,1) {$y=f(x)=\sin x$};
                \node [below] at (-0.5*pi,-2) {$y=g(x)=2\sin x$};
            \end{tikzpicture}
            \caption{Vertical scaling.}
            \label{fig:scalingb}
        \end{subfigure}
        \caption{Change of scale.}
        \label{fig:scaling}
    \end{figure}
    \begin{itemize}
        \item Suppose $g(x)=f(kx)$. Then to transform the graph of $f$ into that of $g$, \dq{compress or stretch the $x$-axis by shrinking (if $k>1$) or stretching (if $k<1$) every interval of length $k$ on the $x$-axis into an interval of length 1}{24}
        \begin{itemize}
            \item Let $g(x)=f(kx)$. If $D_f=[a,b]$ ($f(x)$ is only defined when $a\leq x\leq b$), then $f(kx)$ is only defined when $a\leq kx\leq b$. This implies that $g(x)$ is only defined when $a/k\leq x\leq b/k$; hence, $D_g=[a/k,b/k]$.
            \item Be wary when $k=0$.
        \end{itemize}
        \item Suppose $g(x)=k\cdot f(x)$. Then to transform the graph of $f$ into that of $g$, \dq{stretch the $f$ curve vertically (if $k>1$), or compress it (if $k<1$) [or] change the scale on the $y$-axis so that points labeled $1,2,3,\dots$ for the graph of $f$ are relabled $k,2k,3k,\dots$ for the graph of [$g$]}{25}
        \begin{itemize}
            \item This kind of stretching may cause $R_g$ to differ from $R_f$ by some factor $k$, but it will not affect $D_f$ and $D_g$.
        \end{itemize}
    \end{itemize}
\end{itemize}



\section{Behavior of Functions}
\begin{itemize}
    \item Linear functions (refer to Figure \ref{fig:behavior-linear} throughout the following).
    \begin{figure}[h!]
        \centering
        \begin{tikzpicture}[
            scale=1.6,
            every node/.style={black,text height=1.5ex,text depth=0.25ex}
        ]
            \footnotesize
            \draw [-stealth] (-1,0) -- (2.5,0) node[right]{$x$};
            \draw [-stealth] (0,-0.6) -- (0,3) node[above]{$y$};

            \fill [yly] (0.45,1.35) -- node[below]{$h$} (0.9,1.35) -- node[right]{$3h$} (0.9,2.7) -- cycle;

            \draw [ylx,thick] (-0.2,-0.6) -- (1,3) node[below right]{$\text{slope}=3$};
            \draw [ylx,semithick] (0.45,1.35) -- (0.45,0) node[below]{$x_1$};
            \draw [ylx,semithick] (0.9,1.35) -- (0.9,0) node[below,xshift=6mm]{$x_2=x_1+h$};

            \node [below right] {$O$};
        \end{tikzpicture}
        \caption{Behavior of linear functions.}
        \label{fig:behavior-linear}
    \end{figure}
    \begin{itemize}
        \item For example, let $f(x)=3x$. The graph of $f$ is a straight line through the origin with slope $+3$.
        \item Let $x_1$ be an initial value of $x$, and let $x_2=x_1+h$ be a new value of $x$ obtained by increasing $x_1$ by $h$ units\footnote{Note that $h$ is an alternative notation for $\Delta x$.}. The corresponding increase in $f(x)$ is given by
        \begin{equation*}
            f(x_1+h)-f(x_1) = 3(x_1+h)-3x_1
            = 3h
        \end{equation*}
        Thus, we see that $f$ everywhere changes three times as fast as $x$.
    \end{itemize}
    \item Quadratic functions (refer to Figure \ref{fig:behavior-quadratic} throughout the following).
    \begin{figure}[h!]
        \centering
        \begin{tikzpicture}[
            yscale=0.35,
            every node/.style={black}
        ]
            \footnotesize
            \draw [-stealth] (-4.5,0) -- (4.5,0) node[right]{$x$};
            \draw [-stealth] (0,-2.5) -- (0,16) node[above]{$y$};
            \foreach \x in {-4,-3,-2,-1,1,2,3,4} {
                \draw (\x,0.3) -- ++(0,-0.6) node[below]{$\x$};
            }
            \foreach \y in {4,8,12} {
                \draw (0.1,\y) -- ++(-0.2,0) node[left]{$\y$};
            }

            \draw [ylx,semithick] (-3,6.25) -- (-3,0);
            \draw [ylx,semithick] (-2.5,6.25) -- (-2.5,0);
            \fill [yly] (-3,9) node[right]{$(-3,9)$} -- node[left]{$-6h+h^2$} (-3,6.25) -- node[below]{$h$} (-2.5,6.25) node[right,fill=white,inner sep=2pt]{$(-3+h,9-6h+h^2)$} -- cycle;
            \draw [line width=0.3pt,->] (-2.5,-1.3) node[below]{$-3+h$} -- (-2.5,0);
            
            \draw [ylx,semithick] (3,9) -- (3,0);
            \draw [ylx,semithick] (3.5,9) -- (3.5,0);
            \fill [yly] (3,9) node[left]{$(3,9)$} -- node[below]{$h$} (3.5,9) -- node[right]{$6h+h^2$} (3.5,12.25) node[left]{$(3+h,9+6h+h^2)$} -- cycle;
            \draw [line width=0.3pt,->] (3.5,-1.3) node[below]{$3+h$} -- (3.5,0);

            \draw [ylx,thick] (-4,16) node[right]{$y=f(x)=x^2$} parabola bend (0,0) (4,16);

            \draw ($(-3,9)!-2.3!(-2.5,6.25)$) node[below left,text width=2cm,align=right]{slope of chord $= -6+h$} -- ($(-3,9)!3!(-2.5,6.25)$);
            \draw ($(3,9)!2!(3.5,12.25)$) node[below right,text width=2cm]{slope of chord $= 6+h$} -- ($(3,9)!-2!(3.5,12.25)$);

            \node [below left=1mm] {$0$};
        \end{tikzpicture}
        \caption{Behavior of quadratic functions.}
        \label{fig:behavior-quadratic}
    \end{figure}
    \begin{itemize}
        \item Unlike linear functions, quadratic functions do not everywhere have a constant rate of change.
        \item Let $f(x)=x^2$. In an analogous manner to the previous example, we find that the change between $x_1$ and $x_2=x_1+h$ is
        \begin{equation*}
            f(x_1+h)-f(x_1) = (x_1+h)^2-x_1^2
            = 2x_1h+h^2
        \end{equation*}
        Thus, we see that the rate of change of $f$ is dependent on both the initial value of $x$ and the amount of increase in $x$.
        \item While a linear function increases at a rate directly proportional to the increase in $x$, the above demonstrates that \dq{the increase in $x^2$, as $x$ increases from $x_1$ to $x_1+h$, is $2x_1+h$ times the increase in $x$}{27}
        \item We can now define the \textbf{average rate of increase}.
        \item Thus, the average rate of increase of $x^2$ is $2x_1+h$.
        \item Now let's see what happens as $h$ shrinks.
        \begin{table}[h!]
            \centering
            \setlength{\tabcolsep}{4mm}
            \begin{tabular}{r|l|l|l|l|l}
                \multicolumn{2}{c|}{\multirow{2}{*}{}} & \multicolumn{4}{c}{\rule[-0.2cm]{0pt}{0.6cm} $x_1$}\\
                \cline{3-6}
                 \multicolumn{2}{c|}{} & \multicolumn{1}{c|}{\rule[-0.2cm]{0pt}{0.6cm} 2} & \multicolumn{1}{c|}{3} & \multicolumn{1}{c|}{$-2$} & \multicolumn{1}{c}{$-3$}\\
                \hline
                \multirow{6}{*}{$h$} & 1 & 5 & 7 & $-3$ & $-5$\\
                 & 0.5 & 4.5 & 6.5 & $-3.5$ & $-5.5$\\
                 & 0.25 & 4.25 & 6.25 & $-3.75$ & $-5.75$\\
                 & 0.1 & 4.1 & 6.1 & $-3.9$ & $-5.9$\\
                 & 0.01 & 4.01 & 6.01 & $-3.99$ & $-5.99$\\
                 & 0.001 & 4.001 & 6.001 & $-3.999$ & $-5.999$
            \end{tabular}
            \caption{Average rate of change of $x^2$ versus $h$.}
            \label{tab:2x1+h}
        \end{table}
        \item From Table \ref{tab:2x1+h}, we see that smaller values of $h$ cause the average rate of change to tend toward $2x_1$. This is the beginning of \textbf{differential calculus}.
    \end{itemize}
    \item \textbf{Average rate of increase} (of $f(x)$, per unit of increase in $x$, from $x_1$ to $x_1+h$): The ratio,
    \begin{equation*}
        \frac{f(x_1+h)-f(x_1)}{(x_1+h)-x_1} = \frac{f(x_1+h)-f(x_1)}{h}
        = \frac{\text{change in }f(x)}{\text{change in }x}
    \end{equation*}
    \item \textbf{Differential calculus}: The branch of calculus concerned with the \emph{instantaneous} rate of increase of a function, as opposed to the \emph{average} rate of increase.
    \item So constant and linear functions are easy to analyze. But for more complicated functions, we need more advanced tools.
    \item Let's begin exploring the instantaneous rate of change, continuing with the parabola example.
    \begin{itemize}
        \item For $x^2$, the average rate of change is given by $\frac{f(x_1+h)-f(x_1)}{h}=2x_1+h,\ h\neq 0$.
        \item The $h\neq 0$ is critical --- we wish to consider the case where $h=0$, but we cannot. However, we can consider values of the slope function $m(h)=2x_1+h$ in a deleted neighborhood of $h=0$. By decreasing the radius of the neighborhood, we can get progressively closer to analytically approximating $m(0)$.
        \begin{figure}[h!]
            \centering
            \begin{tikzpicture}[
                scale=1.4,
                every node/.style={black}
            ]
                \footnotesize
                \draw [-stealth] (-1.5,0) -- (2.5,0) node[right]{$h$};
                \draw [-stealth] (0,-0.5) -- (0,3) node[above]{$m(h)$};

                \draw [ylx,thick,dashed,dash pattern=on 6pt off 2pt] (-1.4,-0.4) -- (1.5,2.5) node[below right]{$m(h)=2x_1+h$};

                \draw [ylx,semithick] (-0.5,1.5) -- (-0.5,0) node[below]{$-\epsilon$};
                \draw [ylx,semithick] (0.5,1.5) -- (0.5,0) node[below]{$\epsilon$};

                \draw [ylx,semithick] (-1.5,0.5) -- (2,0.5) node[right]{$2x_1-\epsilon$};
                \draw [ylx,semithick] (-1.5,1) -- (2,1) node[right]{$2x_1$};
                \draw [ylx,semithick] (-1.5,1.5) -- (2,1.5) node[right]{$2x_1+\epsilon$};

                \draw [semithick,<->] (1.8,0.52) -- node[right]{$\epsilon$} (1.8,0.98);
                \draw [semithick,<->] (1.8,1.02) -- node[right]{$\epsilon$} (1.8,1.48);

                \draw [ylx,thick,postaction={decorate},decoration={markings, mark=between positions 0 and 1 step 0.5 with \fill [white] circle (1.5pt);\draw circle (1.5pt);}] (-0.5,0.5) -- (0.5,1.5);

                \node [below left] {$O$};
            \end{tikzpicture}
            \caption{Deleted neighborhood of a slope function.}
            \label{fig:slopeFunction}
        \end{figure}
        \item From Figure \ref{fig:slopeFunction}, we can see that $m(h)$ is bounded between $2x_1+\epsilon$ and $2x_1-\epsilon$ when $|h|$ is less than $\epsilon$ (or any positive number smaller than $\epsilon$)\footnote{Note that $\epsilon$ is used to denote an arbitrary (often arbitrarily small) positive number.}.
        \item In fact, $0<|h|<\epsilon$, or $-\epsilon<h<\epsilon$, directly implies $2x_1-\epsilon<2x_1+h<2x_1+\epsilon$.
        \item We can now formally define \textbf{approximation}, such as what was just described.
    \end{itemize}
    \item \textbf{Approximation} (of $f(x)$ by $L$ to within $\epsilon$ on $(a,b)$): The value $L$ approximates $f(x)$ to within $\epsilon$ on the interval $(a,b)$ if $L-\epsilon<f(x)<L+\epsilon$ when $a<x<b$.
\end{itemize}



\section{Slope of a Curve}
\begin{itemize}
    \item \textbf{Slope of the curve} (at $P$): The limiting value of the slope of the secant between distinct points $P,Q$ on the curve $y=f(x)$ as $Q$ moves along the curve progressively closer to $P$. \emph{Also known as} \textbf{slope of the tangent to the curve} (at $P$).
    \begin{itemize}
        \item A purely geometric definition also exists: \dq{Let $C$ be a curve and $P$ a point on $C$. If there exists a line $L$ through $P$ such that the measure of one of the angles between $L$ and the secand line $PQ$ approaches zero as $Q$ approaches $P$ along $C$, then $L$ is said to be tangent to $C$ at $P$}{30}
        \item An advantage of the geometric definition is that it does not depend on the coordinate axes and allows vertical lines.
        \item However, in most cases, we will stick with the algebraic definition.
    \end{itemize}
    \item \cite{bib:Thomas} considers the average rate of increase equation for a cubic function, informally allowing $\Delta x$ to tend towards 0 to derive a slope function.
\end{itemize}



\section{Derivative of a Function}
\begin{itemize}
    \item We now formalize our notion of a slope function.
    \begin{itemize}
        \item We know that the slope $m_\text{sec}$ of the secant from $P(x,y)$ to a point on the curve $y=f(x)$ at $(x+\Delta x,f(x+\Delta x))$ is given by
        \begin{equation*}
            m_\text{sec} = \frac{f(x+\Delta x)-f(x)}{\Delta x}
        \end{equation*}
        \item Now as $\Delta x$ tends toward 0, $m_\text{sec}$ tends toward the slope $m_\text{tan}$ of the tangent at $P$. The mathematical symbols which summarize this discussion are
        \begin{equation*}
            m_\text{tan} = \lim_{Q\to P}m_\text{sec}
            = \lim_{\Delta x\to 0}\frac{\Delta y}{\Delta x}
            = \lim_{\Delta x\to 0}\frac{f(x+\Delta x)-f(x)}{\Delta x}
        \end{equation*}
    \end{itemize}
    \item The number given by the last operation in the above equation is clearly related to $f$. Thus, to indicate relation, we define
    \begin{equation*}
        \bm{f'(x)} = \lim_{\Delta x\to 0}\frac{f(x+\Delta x)-f(x)}{\Delta x}
    \end{equation*}
    \emph{Also known as} $\bm{y'}$, $\bm{\frac{\mathrm{d}y}{\mathrm{d}x}}$, $\bm{D_xy}$.
    \item This limit may sometimes fail to exist. However, at each point where it does exist, $f$ is said to have a \textbf{derivative}, or to be \textbf{differentiable}. Similarly, $f'(x)$ is said to be the \textbf{derivative} (of $f$ at $x$).
    \item Differential calculus is concerned with two problems.
    \begin{enumerate}
        \item \dq{Given a function $f$, determine those values of $x$ (in the domain of $f$) at which the function possesses a derivative}{32}
        \item \dq{Given a function $f$ and an $x$ at which the derivative exists, find $f'(x)$}{32}
    \end{enumerate}
    \item \textbf{Derived function}: \dq{The set of all pairs of numbers $(x,f'(x))$ that can be formed by this process}{33} \emph{Also known as} \textbf{derivative} (of $f$).
    \begin{itemize}
        \item \dq{The domain of $f'$ is a subset of the domain of $f$}{33}
        \item Symbolically, $D_{f'}\subset D_f$. However, for most functions considered in this book, $D_{f'}=D_f$ with maybe a few exceptions.
    \end{itemize}
    \item On computing $f'(x)$ by eliminating the division by 0 and then substituting: \dq{We may say that after the division by $\Delta x$ has been carried out and the expression has been reduced to a form\dots which `makes sense' (that is, does not involve division by zero) when $\Delta x$ is taken equal to zero, then the limit as $\Delta x$ approaches zero does exist and may be found by simply replacing $\Delta x$ by zero in this reduced form}{34}
    \item Essentially, what we are doing when we eliminate the division by zero is we are expanding the domain of the function, the limit of which we are taking, to include a point of interest (\cite{bib:Thomas} elaborates quite a bit on this point).
\end{itemize}



\section{Velocity and Rates}
\begin{itemize}
    \item Mainly just applies average and instantaneous rates of change to the physical problem of distance and velocity. However\dots
    \item \dq{Derivatives are important in economic theory, where they are usually indicated by the adjective \textbf{marginal}}{37}
    \begin{itemize}
        \item \dq{Suppose that in order to produce $x+\Delta x$ tons of steel weekly, it would cost $y+\Delta y$ dollars. The increase in cost per unit increase in output would be $\Delta y/\Delta x$. The limit of this ratio, as $\Delta x$ tends to zero, is called the \textbf{marginal cost}}{37}
        \item There also exists \textbf{marginal revenue} $\dv*{P}{x}$ and \textbf{marginal profit} $\dv*{T}{x}$.
    \end{itemize}
    \item Note: \dq{The \emph{average} rate of change of $y$ per unit change in $x$, $\Delta y/\Delta x$, when multiplied by the number of units change in $x$, $\Delta x$, gives the actual change in $y$:
    \begin{equation*}
        \Delta y = \frac{\Delta y}{\Delta x}\, \Delta x
    \end{equation*}
    The \emph{instantaneous} rate of change of $y$ per unit change in $x$, $f'(x)$, multiplied by the number of units change in $x$, $\Delta x$, gives the change that would be produced in $y$ if the point $(x,y)$ were to move along the tangent line instead of moving along the curve; that is
    \begin{equation*}
        \Delta y_\text{tan} = f'(x)\, \Delta x
    \end{equation*}
    One reason calculus is important is that it enables us to find quantitatively how a change in one of two related variables affects the second variable}{38}
\end{itemize}




\end{document}