\documentclass[../main.tex]{subfiles}

\pagestyle{main}
\renewcommand{\chaptermark}[1]{\markboth{\chaptername\ \thechapter:\ #1}{}}
\setcounter{chapter}{13}

\begin{document}




\chapter{Vector Functions and Their Derivatives}
\section{Introduction}
\begin{itemize}
    \item \marginnote{12/14:}\textbf{Vector function} (of $h$): A function $\textbf{F}(h)$ with $n$ components where each component is a function. Essentially, $\textbf{F}=(f_1,f_2,\dots,f_n)$.
    \item \textbf{Limit} (of $\textbf{F}(h)$ as $h\to a$): If each component $f_1,\dots,f_n$ of $\textbf{F}$ has a limit $L_1,\dots,L_n$ as $h\to a$, then
    \begin{equation*}
        \lim_{h\to a}\textbf{F}(h) = (L_1,\dots,L_n)
    \end{equation*}
    \item \textbf{Continuous} (vector function $\textbf{F}$ at $a$): A vector function $\textbf{F}$ where for every $\epsilon>0$, there corresponds a $\delta>0$ such that
    \begin{equation*}
        |\textbf{F}(h)-\textbf{F}(a)|<\epsilon\quad\text{when}\quad|h-a|<\delta
    \end{equation*}
    \begin{itemize}
        \item \cite{bib:Thomas} shows that this is equivalent to the requirement that each component of $\textbf{F}$ is continuous at $a$.
    \end{itemize}
    \item \textbf{Derivative} (of a vector function at $c$): The derivative $\textbf{F}'(c)$ of a vector function $\textbf{F}$ at $c$ is given by the equation
    \begin{equation*}
        \textbf{F}'(c) = \lim_{h\to 0}\frac{\textbf{F}(c+h)-\textbf{F}(c)}{h}
    \end{equation*}
    \begin{itemize}
        \item It can be proven that $\textbf{F}$ is differentiable at $c$ if and only if each of its components are differentiable at $c$, and that if this condition is met,
    \end{itemize}
    \begin{equation*}
        \textbf{F}'(c) = (f_1'(c),\dots,f_n'(c))
    \end{equation*}
\end{itemize}



\section{Velocity and Acceleration}
\begin{itemize}
    \item Results from here on out will generally pertain to 2D questions, but these methods can easily be generalized to higher dimensions.
    \item Applications of vectors to physics problems.
    \begin{itemize}
        \item To solve \textbf{statics} problems, we only need to know the \textbf{algebra} of vectors.
        \item To solve \textbf{dynamics} problems, we also need to know the \textbf{calculus} of vectors.
    \end{itemize}
    \item \textbf{Position vector}: The vector from the origin to a point $P$ that moves along a parametrically defined curve. \emph{Denoted by} $\textbf{R}$.
    \item \textbf{Velocity vector}: The vector tangent to a point $P$ that moves along a parametrically defined curve and with magnitude $|\text{d}s/\text{d}t|$. \emph{Denoted by} $\textbf{v}$.
    \begin{figure}[H]
        \centering
        \begin{tikzpicture}[
            every node/.append style={black}
        ]
            \footnotesize
            \draw [->] (0,-1) -- (0,3.5) node[above]{$y$};
            \draw [->] (-0.6,0) -- (5,0) node[right]{$x$};
            \node [anchor=north east] {$O$};

            \draw [ylx,thick] (0.3,-0.5)
                to[out=-45,in=180] (1,-0.8)
                to[out=0,in=-105,in looseness=1.2] node[below right]{$s$} (2.8,0.9)
                to[out=75,in=180,in looseness=0.8] node[above left]{$\Delta s$} (4.2,3.3)
                to[out=0,in=140] (4.7,3.1)
            ;

            \draw [yly,thick,-latex] (2.8,0.9) -- node[near end,below]{$\textbf{i}\Delta x$} (4.2,0.9);
            \draw [yly,thick,-latex] (4.2,0.9) -- node[near end,right]{$\textbf{j}\Delta y$} (4.2,3.3);
            \draw [ylx,thick,-latex] (0,0) -- node[pos=0.85,below]{$\textbf{R}$} (2.8,0.9) node[above left]{$P(x,y)$};
            \draw [ylx,thick,-latex] (2.8,0.9) -- node[pos=0.4,right]{$\Delta\textbf{R}$} (4.2,3.3) node[above]{$Q$};

            \draw (1,-0.7) node[above]{$(x_0,y_0)$} -- ++(0,-0.2) node[below]{$
                \begin{aligned}
                    t &= 0\\[-3pt]
                    s &= 0
                \end{aligned}
            $};
        \end{tikzpicture}
        \caption{Velocity vector.}
        \label{fig:velocityVector}
    \end{figure}
    \begin{itemize}
        \item \cite{bib:Thomas} semi-rigorously proves from Figure \ref{fig:velocityVector} that if $\textbf{R}$ is the position vector, then $\dv*{\textbf{R}}{t}$ is the velocity vector.
        \item Essentially, he proves that
        \begin{equation*}
            \dv{\textbf{R}}{t} = \textbf{i}\dv{x}{t}+\textbf{j}\dv{y}{t}
        \end{equation*}
        It follows from this that
        \begin{align*}
            \text{slope of }\dv{\textbf{R}}{t} &= \frac{\textbf{j}\text{-component}}{\textbf{i}\text{-component}}
            = \frac{\dv*{y}{t}}{\dv*{x}{t}}
            = \dv{y}{x}\\
            \left| \dv{R}{t} \right| &= \left| \textbf{i}\dv{x}{t}+\textbf{j}\dv{y}{t} \right|
            = \sqrt{\left( \dv{x}{t} \right)^2+\left( \dv{y}{t} \right)^2}
            = \left| \dv{s}{t} \right|
        \end{align*}
    \end{itemize}
    \item \textbf{Acceleration vector}: The derivative of the velocity vector and second derivative of the position vector. \emph{Denoted by} $\textbf{a}$.
    \begin{equation*}
        \textbf{a} = \dv{\textbf{v}}{t} = \textbf{i}\dv[2]{x}{t}+\textbf{j}\dv[2]{y}{t}
    \end{equation*}
    \item Sometimes, we are given a force vector $\textbf{F}=m\textbf{a}$ and initial conditions.
    \begin{itemize}
        \item From these, we can solve for velocity and position vectors via fairly straightforward component integration.
        \item Note, however, that constants of integration are now vectors.
    \end{itemize}
\end{itemize}



\section{Tangential Vectors}
\begin{itemize}
    \item Let $P_0$ be a point on a curve. The distance $s$ from $P_0$ to some point $P$ along the curve is clearly related to the position of $P$. Thus, we may think of $\textbf{R}$ as a function of $s$, and investigate the properties of $\dv*{\textbf{R}}{s}$.
    \item \textbf{Tangent vector}: The unit vector tangent to a point $P$ along a curve.
    \begin{figure}[h!]
        \centering
        \begin{subfigure}[b]{0.33\linewidth}
            \centering
            \begin{tikzpicture}[
                every node/.append style={black}
            ]
                \footnotesize
                \draw [->] (-0.5,0) -- (4,0) node[right]{$x$};
                \draw [->] (0,-0.7) -- (0,3) node[above]{$y$};
                \node [anchor=north east] {$O$};

                \draw [yly,thick,-latex] (2.4,0.52) -- node[near end,below]{$\textbf{i}\Delta x$} (3.3,0.52);
                \draw [yly,thick,-latex] (3.3,0.52) -- node[near end,right]{$\textbf{j}\Delta y$} (3.3,2.36);
                \draw [ylx,thick,-latex] (2.4,0.52) -- node[very near end,left]{$\Delta\textbf{R}$} (3.3,2.36) node[above left]{$Q$};
                \draw [ylx,thick,-latex] (0,0) -- node[pos=0.7,above]{$\textbf{R}$} (2.4,0.52) node[above left]{$P$};

                \draw [ylx,thick,decoration={markings,mark=at position 0.5 with {\node[below,xshift=-4mm,yshift=-6mm]{$s$};},mark=at position 0.65 with {\node[right]{$\Delta s$};}},postaction={decorate}] plot[domain=0.1:3.5,smooth] (\x,{0.17*\x^2.5-1});

                \draw (0.95,-0.7) node[above]{$P_0$} -- (1.05,-0.96) node[below]{$s=0$};
            \end{tikzpicture}
            \caption{$\Delta s>0$.}
            \label{fig:tangentVectora}
        \end{subfigure}
        \begin{subfigure}[b]{0.32\linewidth}
            \centering
            \begin{tikzpicture}[
                every node/.append style={black}
            ]
                \footnotesize
                \draw [->] (-0.5,0) -- (4,0) node[right]{$x$};
                \draw [->] (0,-0.7) -- (0,3) node[above]{$y$};
                \node [anchor=north east] {$O$};

                \draw [yly,thick,-latex] (2.4,0.52) -- node[near end,right]{$\textbf{j}\Delta y$} (2.4,-0.45);
                \draw [yly,thick,-latex] (2.4,-0.45) -- node[pos=0.4,below]{$\textbf{i}\Delta x$} (1.6,-0.45);
                \draw [ylx,thick,-latex] (2.4,0.52) -- node[near end,left=1mm]{$\Delta\textbf{R}$} (1.6,-0.45) node[below]{$Q$};
                \draw [ylx,thick,-latex] (0,0) -- node[pos=0.7,above]{$\textbf{R}$} (2.4,0.52) node[above left]{$P$};

                \draw [ylx,thick,decoration={markings,mark=at position 0.5 with {\node[right,xshift=-5mm,yshift=-6mm]{$\Delta s$};}},postaction={decorate}] plot[domain=0.1:3.5,smooth] (\x,{0.17*\x^2.5-1});

                \draw (0.95,-0.7) node[above]{$P_0$} -- (1.05,-0.96) node[below]{$s=0$};
            \end{tikzpicture}
            \caption{$\Delta s<0$.}
            \label{fig:tangentVectorb}
        \end{subfigure}
        \begin{subfigure}[b]{0.33\linewidth}
            \centering
            \begin{tikzpicture}[
                every node/.append style={black}
            ]
                \footnotesize
                \draw [->] (-0.5,0) -- (4,0) node[right]{$x$};
                \draw [->] (0,-0.7) -- (0,3) node[above]{$y$};
                \node [anchor=north east] {$O$};

                \draw [ylx,thick,-latex] (2.4,0.52) -- node[right]{$\textbf{T}$} (3.3,1.95);
                \draw [ylx,thick,-latex] (0,0) -- node[pos=0.7,above]{$\textbf{R}$} (2.4,0.52) node[above left]{$P$};

                \draw [ylx,thick,decoration={markings,mark=at position 0.5 with {\node[below,xshift=-4mm,yshift=-6mm]{$s$};}},postaction={decorate}] plot[domain=0.1:3.5,smooth] (\x,{0.17*\x^2.5-1});

                \draw (0.95,-0.7) node[above]{$P_0$} -- (1.05,-0.96) node[below]{$s=0$};
            \end{tikzpicture}
            \caption{$\Delta s\to 0$.}
            \label{fig:tangentVectorc}
        \end{subfigure}
        \caption{Tangent vector.}
        \label{fig:tangentVector}
    \end{figure}
    \begin{itemize}
        \item Since $\Delta\textbf{R}$ and $\Delta s$ approach the same quantity as $\Delta s\to 0$, $\Delta\textbf{R}/\Delta s$ approaches unity, i.e., $|\text{d}\textbf{R}/\text{d}s|=1$.
        \item Because of the sign change, whether $\Delta s$ is positive or negative, $\Delta\textbf{R}/\Delta s$ points in the same general direction for sufficiently small $\Delta s$. Indeed, it converges to pointing tangentially.
        \item Thus,
        \begin{equation*}
            \textbf{T} = \dv{\textbf{R}}{s} = \textbf{i}\dv{x}{s}+\textbf{j}\dv{y}{s}
        \end{equation*}
    \end{itemize}
    \item There are two different ways to find $\textbf{T}$: Straight differentiation combined with manipulations of differentials, and the chain rule combined with the dot product. We will explore each, in turn, with an example.
    \item \dq{Find the unit vector $\textbf{T}$ tangent to the circle $x=a\cos\theta$, $y=a\sin\theta$ at any point $P(x,y)$}{471}
    \begin{itemize}
        \item From the given equations, we have
        \begin{align*}
            \dd{x} &= -a\sin\theta\dd{\theta}&
                \dd{y} &= a\cos\theta\dd{\theta}&
                    \dd{s}^2 &= \dd{x}^2+\dd{y}^2\\
            &&
                &&
                    &= a^2(\sin^2\theta+\cos^2\theta)\dd{\theta}^2\\
            &&
                &&
                    &= a^2\dd{\theta}^2\\
            &&
                &&
                    \dd{s} &= \pm a\dd{\theta}
        \end{align*}
        \begin{itemize}
            \item We could alternatively obtain $\dd{s}$ by expressing the arc length formula $S=R\theta$ in terms of differentials.
        \end{itemize}
        \item \dq{If we measure arc length in the counterclockwise direction, with $s=0$ at $(a,0)$, $s$ will be an increasing function of $\theta$, so the $+$-sign should be taken: $\dd{s}=a\dd{\theta}$}{471}
        \item Therefore,
        \begin{align*}
            \textbf{T} &= \textbf{i}\dv{x}{s}+\textbf{j}\dv{y}{s}\\
            &= \textbf{i}\left( \frac{-a\sin\theta\dd{\theta}}{a\dd{\theta}} \right)+\textbf{j}\left( \frac{a\cos\theta\dd{\theta}}{a\dd{\theta}} \right)\\
            &= -\textbf{i}\sin\theta+\textbf{j}\cos\theta
        \end{align*}
    \end{itemize}
    \item The equations
    \begin{align*}
        x &= a\cos\omega t&
            y &= a\sin\omega t&
                z &= bt
    \end{align*}
    where $a,b,\omega$ are positive constants define a circular helix in $E^3$\footnote{Three-dimensional Euclidean space, equivalent to $\mathbb{R}^3$}.
    \begin{itemize}
        \item Let $P_0=(a,0,0)$, since this is the point on the locus of the parametric equations where $t=0$. Additionally, let arc length be measured in the direction in which $P$ moves away from $P_0$ as $t$ increases from 0.
        \item Using the chain rule to differentiate, we have
        \begin{align*}
            \textbf{T} &= \textbf{i}\dv{x}{s}+\textbf{j}\dv{y}{s}+\textbf{k}\dv{z}{s}\\
            &= \textbf{i}\left( -a\omega\sin\omega t\, \dv{t}{s} \right)+\textbf{j}\left( a\omega\cos\omega t\, \dv{t}{s} \right)+\textbf{k}\left( b\, \dv{t}{s} \right)
        \end{align*}
        \item Since $\textbf{T}$ is a unit vector, we have $1=|\textbf{T}|=|\textbf{T}|^2=\textbf{T}\cdot\textbf{T}$. Thus,
        \begin{align*}
            1 &= \textbf{T}\cdot\textbf{T}\\
            &= \textbf{i}\cdot\textbf{i}\left( -a\omega\sin\omega t\, \dv{t}{s} \right)^2+\textbf{j}\cdot\textbf{j}\left( a\omega\cos\omega t\, \dv{t}{s} \right)^2+\textbf{k}\cdot\textbf{k}\left( b\, \dv{t}{s} \right)^2\\
            % &= 1\left( a^2\omega^2\sin^2\omega t\left( \dv{t}{s} \right)^2 \right)+1\left( a^2\omega^2\cos^2\omega t\left( \dv{t}{s} \right)^2 \right)+1\left( b^2\left( \dv{t}{s} \right)^2 \right)\\
            % &=\left( a^2\omega^2\sin^2\omega t+a^2\omega^2\cos^2\omega t+b^2 \right)\left( \dv{t}{s} \right)^2\\
            % &=\left( a^2\omega^2\left( \sin^2\omega t+\cos^2\omega t \right)+b^2 \right)\left( \dv{t}{s} \right)^2\\
            &=\left( a^2\omega^2+b^2 \right)\left( \dv{t}{s} \right)^2\\
            \dv{t}{s} &= \pm\frac{1}{\sqrt{a^2\omega^2+b^2}}
        \end{align*}
        \item We choose the $+$-sign because $s$ should be a positive function of $t$.
        \item Putting this all together, we get
        \begin{equation*}
            \textbf{T} = \frac{a\omega(-\textbf{i}\sin\omega t+\textbf{j}\cos\omega t)+\textbf{k}b}{\sqrt{a^2\omega^2+b^2}}
        \end{equation*}
    \end{itemize}
\end{itemize}



\section{Curvature and Normal Vectors}
\begin{itemize}
    \item \marginnote{12/15:}\textbf{Curvature}: The rate of change of the slope angle $\phi$ between $\textbf{T}$ and the $x$-axis with respect to the arc length $s$. \emph{Denoted by} $\kappa$.
    \begin{figure}[h!]
        \centering
        \begin{tikzpicture}[
            every node/.append style={black}
        ]
            \footnotesize
            \draw [->] (-0.5,0) -- (4,0) node[right]{$x$};
            \draw [->] (0,-0.5) -- (0,3.5) node[above]{$y$};
            \node [anchor=north east] {$O$};

            \coordinate (P) at (1.8,0.99);
            \coordinate (A) at (3.5,0.99);
            \coordinate (B) at (3,1.67);
            \draw (P) -- (A);
            \draw [yly,thick,-latex] (P) node[above left]{$P$} -- (B) node[right]{$\textbf{T}=\dv{\textbf{R}}{s}$};
            \draw [ylx,thick] plot[domain=0.1:3.8,smooth] (\x,{0.2*(\x-0.4)^2+0.6});
            \pic [draw,->,angle radius=8mm,angle eccentricity=1.2,pic text={$\phi$}] {angle=A--P--B};
            
            \draw (0.4,0.7) -- ++(0,-0.2) node[below]{$P_0$};
            \node at (1.2,0.6) {$s$};
        \end{tikzpicture}
        \caption{Curvature.}
        \label{fig:curvature}
    \end{figure}
    \begin{itemize}
        \item Measured in radians per unit length.
    \end{itemize}
    \item From the facts that
    \begin{align*}
        \kappa &= \dv{\phi}{s}&
            \tan\phi &= \dv{y}{x}&
                \dd{s} &= \pm\sqrt{\dd{x}^2+\dd{y}^2}
    \end{align*}
    we can derive a formula for $\kappa$ in terms of the original function $y=f(x)$ as follows.
    \begin{align*}
        \phi &= \tan^{-1}\dv{y}{x}&
            \dv{s}{x} &= \pm\sqrt{1+\left( \dv{y}{x} \right)^2}\\
        \dv{\phi}{x} &= \frac{\dv[2]{y}{x}}{1+\left( \dv{y}{x} \right)^2}
    \end{align*}
    \begin{align*}
        \kappa &= \left| \dv{\phi}{s} \right|
        = \left| \frac{\dv*{\phi}{x}}{\dv*{s}{x}} \right|\\
        &= \frac{\left| \dv[2]{y}{x} \right|}{\left[ 1+\left( \dv{y}{x} \right)^2 \right]^{3/2}}
    \end{align*}
    \item We can similarly derive that
    \begin{equation*}
        \kappa = \frac{\left| \dv[2]{x}{y} \right|}{\left[ 1+\left( \dv{x}{y} \right)^2 \right]^{3/2}}
    \end{equation*}
    \item If the equations for $y$ and $x$ are given parametrically in terms of $t$, we have
    \begin{align*}
        \kappa &= \frac{\left| \dv{x}{t}\dv[2]{y}{t}-\dv{y}{t}\dv[2]{x}{t} \right|}{\left[ \left( \dv{x}{t} \right)^2+\left( \dv{y}{t} \right)^2 \right]^{3/2}}\\
        &= \frac{|\dot{x}\ddot{y}-\dot{y}\ddot{x}|}{\left[ \dot{x}^2+\dot{y}^2 \right]^{3/2}}
    \end{align*}
    \item Naturally, the curvature of a straight line should be 0. Indeed, we find this from the above equations.
    \item Naturally, the curvature of a circle should be constant, and should somehow decrease as the radius increases. Indeed, we find from the facts that $s=r\theta$ and $\phi=\theta+\frac{\pi}{2}$ that
    \begin{equation*}
        \kappa = \left| \dv{\phi}{s} \right|
        = \left| \frac{\dd{\theta}}{r\dd{\theta}} \right|
        = \frac{1}{r}
    \end{equation*}
    \begin{figure}[h!]
        \centering
        \begin{tikzpicture}[
            scale=1.6,
            every node/.append style={black}
        ]
            \footnotesize
            \draw [yly,thick] (-0.5,1.25) circle ({2^0.5});
            \draw [ylx,semithick,decoration={markings,mark=at position 0 with {\fill circle (1.5pt);}},postaction={decorate}] (-0.5,1.25) node[above,text width=2cm,align=center]{center of curvature} -- node[pos=0.25,right=-2mm,text width=2cm,align=center]{radius of curvature} (0.5,0.25);
            \draw [ylx,thick,decoration={markings,mark=at position 0.9 with {\node[right]{curve};}},postaction={decorate}] plot[domain=-1.3:1.3,smooth] (\x,{\x*\x});
            \node [text width=2cm,align=center] at (0.8,2.4) {osculating circle};

            \draw (0,-0.25) -- node[below right]{$P(x,y)$} (1,0.75);
        \end{tikzpicture}
        \caption{Circle, radius, and center of curvature.}
        \label{fig:circleCurvature}
    \end{figure}
    \item \textbf{Circle of curvature} (at $P$): \dq{The circle that is tangent to a given curve at $P$, whose center lies on the concave side of the curve and which has the same curvature as the curve has at $P$}{475} \emph{Also known as} \textbf{osculating circle}.
    \begin{itemize}
        \item Calling the circle of curvature the "osculating circle" refers to the fact that its first and second derivatives at $P$ are equal to the first and second derivatives of the curve at $P$, meaning that it has a higher degree of contact with the curve at $P$ than any other circle.
    \end{itemize}
    \item \textbf{Radius of curvature} (at $P$): The radius of the circle of curvature at $P$. \emph{Denoted by} $\rho$.
    \begin{equation*}
        \rho = \frac{1}{\kappa}
        = \frac{\left[ 1+\left( \dv{y}{x} \right)^2 \right]^{3/2}}{\left| \dv[2]{y}{x} \right|}
    \end{equation*}
    \item \textbf{Normal vector}: The unit vector normal to a point $P$ along a curve.
    \begin{itemize}
        \item Observe that $\textbf{T}$ can be expressed in terms of the slope angle $\phi$:
        \begin{equation*}
            \textbf{T} = \textbf{i}\cos\phi+\textbf{j}\sin\phi
        \end{equation*}
        \item Since $\textbf{T}$ can be thought of as a function of $\phi$, we can investigate the properties of $\dv*{\textbf{T}}{\phi}$.
        \item Indeed, it is not difficult to show that
        \begin{align*}
            \dv{\textbf{T}}{\phi} &= -\textbf{i}\sin\phi+\textbf{j}\cos\phi&
                \left| \dv{\textbf{T}}{\phi} \right| &= \sqrt{\sin^2\phi+\cos^2\phi} = 1&
                    \textbf{T}\cdot\dv{\textbf{T}}{\phi} &= 0
        \end{align*}
        \item Thus,
        \begin{equation*}
            \textbf{N} = \dv{\textbf{T}}{\phi} = -\textbf{i}\sin\phi+\textbf{j}\cos\phi
        \end{equation*}
    \end{itemize}
    \item In 3D space, it is harder to define a single normal vector, so we define the\dots
    \item \textbf{Principal normal vector}: The vector
    \begin{equation*}
        \textbf{N} = \frac{\dv*{\textbf{T}}{s}}{|\text{d}\textbf{T}/\text{d}s|}
    \end{equation*}
    \begin{itemize}
        \item We will soon prove that $\textbf{T}\cdot\dv*{\textbf{T}}{s}=0$.
        \item If $\phi$ is an increasing function of $s$, then by the chain rule,
        \begin{equation*}
            \dv{\textbf{T}}{s} = \dv{\textbf{T}}{\phi}\dv{\phi}{s} = \textbf{N}\kappa
        \end{equation*}
        \item Since $\textbf{N}$ is a unit vector and $\kappa$ is a constant, $\kappa$ is equal to the magnitude of $\dv*{\textbf{T}}{s}$.
        \item Thus, we can define the principal normal as above.
    \end{itemize}
    \item \cite{bib:Thomas} uses $\dv*{\textbf{T}}{s}$ to find both the curvature and principal normal vector of the general circular helix investigated earlier. He also checks limiting cases to rederive the curvature of a circle and of a straight line.
    \item \textbf{Binormal vector}: The vector perpendicular to both $\textbf{T}$ and $\textbf{N}$, as defined by
    \begin{equation*}
        \textbf{B} = \textbf{T}\times\textbf{N}
    \end{equation*}
\end{itemize}



\section{Differentiation of Products of Vectors}
\begin{itemize}
    \item Let $\textbf{U}$ and $\textbf{V}$ be vectors whose components are differentiable functions of $t$.
    \item Then we can verify by components that
    \begin{align*}
        \dv{t}(\textbf{U}\cdot\textbf{V}) &= \dv{\textbf{U}}{t}\cdot\textbf{V}+\textbf{U}\cdot\dv{\textbf{V}}{t}&
            \dv{t}(\textbf{U}\times\textbf{V}) &= \dv{\textbf{U}}{t}\times\textbf{V}+\textbf{U}\times\dv{\textbf{V}}{t}
    \end{align*}
    \begin{itemize}
        \item Note that we can \emph{derive} the above, too, through the $\Delta$-process.
        \item Let $\textbf{W}=\textbf{U}\times\textbf{V}$. Then
        \begin{align*}
            \textbf{W}+\Delta\textbf{W} &= (\textbf{U}+\Delta\textbf{U})\times(\textbf{V}+\Delta\textbf{V})\\
            &= \textbf{U}\times\textbf{V}+\textbf{U}\times\Delta\textbf{V}+\Delta\textbf{U}\times\textbf{V}+\Delta\textbf{U}\times\Delta\textbf{V}\\
            \Delta\textbf{W} &= \textbf{U}\times\Delta\textbf{V}+\Delta\textbf{U}\times\textbf{V}+\Delta\textbf{U}\times\Delta\textbf{V}\\
            \frac{\Delta\textbf{W}}{\Delta t} &= \textbf{U}\times\frac{\Delta\textbf{V}}{\Delta t}+\frac{\Delta\textbf{U}}{\Delta t}\times\textbf{V}+\frac{\Delta\textbf{U}}{\Delta t}\times\Delta\textbf{V}\\
            \dv{\textbf{W}}{t} &= \textbf{U}\times\dv{\textbf{V}}{t}+\dv{\textbf{U}}{t}\times\textbf{V}
        \end{align*}
    \end{itemize}
    \item Differentiating the triple scalar product:
    \begin{equation*}
        \dv{t}(\textbf{U}\cdot\textbf{V}\times\textbf{W}) = \dv{\textbf{U}}{t}\cdot\textbf{V}\times\textbf{W}+\textbf{U}\cdot\dv{\textbf{V}}{t}\times\textbf{W}+\textbf{U}\cdot\textbf{V}\times\dv{\textbf{W}}{t}
    \end{equation*}
    \begin{itemize}
        \item Equivalently,
    \end{itemize}
    \begin{equation*}
        \dv{t}
        \begin{vmatrix}
            u_1 & u_2 & u_3\\
            v_1 & v_2 & v_3\\
            w_1 & w_2 & w_3\\
        \end{vmatrix}
        =
        \begin{vmatrix}
            \dv{u_1}{t} & \dv{u_2}{t} & \dv{u_3}{t}\\
            v_1 & v_2 & v_3\\
            w_1 & w_2 & w_3\\
        \end{vmatrix}
        +
        \begin{vmatrix}
            u_1 & u_2 & u_3\\
            \dv{v_1}{t} & \dv{v_2}{t} & \dv{v_3}{t}\\
            w_1 & w_2 & w_3\\
        \end{vmatrix}
        +
        \begin{vmatrix}
            u_1 & u_2 & u_3\\
            v_1 & v_2 & v_3\\
            \dv{w_1}{t} & \dv{w_2}{t} & \dv{w_3}{t}\\
        \end{vmatrix}
    \end{equation*}
    \item Differentiating $\textbf{V}\cdot\textbf{V}=|\textbf{V}|^2$ gives
    \begin{align*}
        \textbf{V}\cdot\dv{\textbf{V}}{t}+\dv{\textbf{V}}{t}\cdot\textbf{V} &= 0\\
        2\textbf{V}\cdot\dv{\textbf{V}}{t} &= 0
    \end{align*}
    \begin{itemize}
        \item Thus, for any vector function $\textbf{V}$, there are three cases: (1) $\textbf{V}=\textbf{0}$, (2) $\dv*{\textbf{V}}{t}=\textbf{0}$, so $\textbf{V}$ is constant in both direction and magnitude, and (3) $\textbf{V}$ and $\dv*{\textbf{V}}{t}$ are perpendicular.
        \item Note that this fact allows to verify that $\textbf{T}\cdot\dv*{\textbf{T}}{s}=0$.
    \end{itemize}
    \item We can use the calculus of tangential and normal vectors to break velocity and acceleration vectors into tangential and normal components.
    \begin{align*}
        \textbf{v} &= \dv{\textbf{R}}{t}&
            \textbf{a} &= \dv{\textbf{v}}{t}&
                \textbf{a} &= \textbf{T}\dv[2]{s}{t}+\dv{s}{t}\dv{\textbf{T}}{s}\dv{s}{t}\\
        &= \dv{\textbf{R}}{s}\dv{s}{t}&
            &= \textbf{T}\dv[2]{s}{t}+\dv{s}{t}\dv{\textbf{T}}{t}&
                &= \textbf{T}\dv[2]{s}{t}+\textbf{N}\kappa\left( \dv{s}{t} \right)^2\\
        &= \textbf{T}\dv{s}{t}&
            &&
                &= \textbf{T}\dv[2]{s}{t}+\textbf{N}\left( \frac{\textbf{v}^2}{\rho} \right)
    \end{align*}
    \begin{itemize}
        \item $\textbf{v}^2/\rho$ is very similar to $v^2/r$ (think about the circle and radius of curvature)!
        \item Another important related equation:
        \begin{equation*}
            |\textbf{a}| = a_T^2+a_N^2
        \end{equation*}
        \item Lastly, we can derive a formula for the curvature in terms of velocity and acceleration.
        \begin{align*}
            \textbf{v}\times\textbf{a} &= \textbf{T}\dv{s}{t}\times\left[ \textbf{T}\dv[2]{s}{t}+\textbf{N}\kappa\left( \dv{s}{t} \right)^2 \right]\\
            &= \textbf{T}\times\textbf{N}\kappa\left( \dv{s}{t} \right)^3\\
            |\textbf{v}\times\textbf{a}| &= \left| \textbf{B}\kappa|\textbf{v}|^3 \right|\\
            \kappa &= \frac{|\textbf{v}\times\textbf{a}|}{|\textbf{v}|^3}
        \end{align*}
    \end{itemize}
\end{itemize}



\section{Polar and Cylindrical Coordinates}
\begin{figure}[h!]
    \centering
    \begin{tikzpicture}[
        every node/.append style={black}
    ]
        \footnotesize
        \draw [->] (-0.5,0) -- (4,0) coordinate (X) node[right]{$x$};
        \draw [->] (0,-0.6) -- (0,3) node[above]{$y$};
        \node [anchor=north east] {$O$};

        \draw [ylx,thick,-latex] (0,0) coordinate (O) -- node[above]{$r$} node[pos=0.82,above]{$\textbf{R}$} (2.7,1) coordinate (P);
        \draw [yly,thick,-latex] (P) -- node[near end,below]{$\textbf{u}_r$} ($(O)!1.4!(P)$) coordinate (ur);
        \draw [yly,thick,-latex] (P) -- node[near end,left]{$\textbf{u}_\theta$} ($(P)!1!90:(ur)$);
        \draw [ylx,thick] (1.3,-0.5) to[out=30,in=-110,in looseness=0.6] (P) node[below right]{$P$} to[out=70,in=-80,out looseness=0.4,in looseness=1.3] (2.8,2.9);
        \pic [draw,->,angle radius=1cm,pic text={$\theta$},angle eccentricity=1.2] {angle=X--O--P};
    \end{tikzpicture}
    \caption{Vectors in polar coordinates.}
    \label{fig:polarVectors}
\end{figure}
\begin{itemize}
    \item To analyze polar coordinates, we introduce the unit vectors
    \begin{align*}
        \textbf{u}_r &= \textbf{i}\cos\theta+\textbf{j}\sin\theta&
            \textbf{u}_\theta &= -\textbf{i}\sin\theta+\textbf{j}\cos\theta
    \end{align*}
    \item Clearly, we have
    \begin{align*}
        \dv{\textbf{u}_r}{\theta} &= \textbf{u}_\theta&
            \dv{\textbf{u}_\theta}{\theta} &= -\textbf{u}_r
    \end{align*}
    \item Additionally, we can see that
    \begin{equation*}
        \textbf{R} = r\textbf{u}_r
    \end{equation*}
    \item The velocity vector can easily be expressed in terms of these quantities (and visualized as such geometrically, as in Figure \ref{fig:polarVelocityVector}).
    \begin{figure}[h!]
        \centering
        \begin{tikzpicture}[
            every node/.append style={black}
        ]
            \footnotesize
            \draw [->] (-0.5,0) -- (4,0) coordinate (X) node[right]{$x$};
            \draw [->] (0,-0.6) -- (0,3) node[above]{$y$};
            \node [anchor=north east] {$O$};
    
            \draw [ylx,thick,-latex] (0,0) coordinate (O) -- node[above]{$r$} node[pos=0.82,above]{$\textbf{R}$} (2.7,1) coordinate (P);
            \draw [yly,thick,-latex] (P) -- node[pos=0.8,below]{$\textbf{u}_r\dv{r}{t}$} (3.81,1.41);
            \draw [yly,thick,-latex] (3.81,1.41) -- node[pos=0.8,right]{$r\dv{\theta}{t}\textbf{u}_\theta$} (3.3,2.8);
            \draw [ylx,thick,-latex] (P) -- node[pos=0.35,right]{$\textbf{T}\dv{s}{t}$} (3.3,2.8);
            \draw [ylx,thick] (1.3,-0.5) to[out=30,in=-110,in looseness=0.6] (P) node[below right]{$P$} to[out=70,in=-80,out looseness=0.4,in looseness=1.3] (2.8,2.9);
            \pic [draw,->,angle radius=1cm,pic text={$\theta$},angle eccentricity=1.2] {angle=X--O--P};
        \end{tikzpicture}
        \caption{Polar velocity vector.}
        \label{fig:polarVelocityVector}
    \end{figure}
    \begin{align*}
        \textbf{v} &= \dv{\textbf{R}}{t}\\
        &= \textbf{u}_r\dv{r}{t}+r\dv{\textbf{u}_r}{t}\\
        &= \textbf{u}_r\dv{r}{t}+r\textbf{u}_\theta\dv{\theta}{t}
    \end{align*}
    \item The acceleration vector can also be expressed in terms of these quantities (the following can be derived by differentiating the above with respect to $t$ and substituting).
    \begin{equation*}
        \textbf{a} = \textbf{u}_r\left[ \dv[2]{r}{t}-r\left( \dv{\theta}{t} \right)^2 \right]+\textbf{u}_\theta\left[ r\dv[2]{\theta}{t}+2\dv{r}{t}\dv{\theta}{t} \right]
    \end{equation*}
    \item In three dimensions (esp. for cylindrical coordinates), we have
    \begin{align*}
        \textbf{R} &= r\textbf{u}_r+\textbf{k}z\\
        \textbf{v} &= \textbf{u}_r\dv{r}{t}+r\textbf{u}_\theta\dv{\theta}{t}+\textbf{k}\dv{z}{t}\\
        \textbf{a} &= \textbf{u}_r\left[ \dv[2]{r}{t}-r\left( \dv{\theta}{t} \right)^2 \right]+\textbf{u}_\theta\left[ r\dv[2]{\theta}{t}+2\dv{r}{t}\dv{\theta}{t} \right]+\textbf{k}\dv[2]{z}{t}
    \end{align*}
    \item \cite{bib:Thomas} goes into an lengthy application of the above definitions to deriving Kepler's Laws.
\end{itemize}




\end{document}